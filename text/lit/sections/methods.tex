% !TEX root = ../lit.tex


\section{Methods best practice}


\subsubsection*{Experimental best practices}

\begin{itemize}

    \item Have two control groups to test whether trajectories are indeed
        identical without treatment.

    \item Do replication within trial: split sample into two and run two
        similar / identical trials \citep{hershfield2019temporal}

    \item If we can identify customers who have added all accounts, can use
        that sample to check where additionally saved money comes from (e.g. do
        more savings lead to fewer purchases of gym classes or sth?)

    \item Look at
        \href{https://blogs.worldbank.org/impactevaluations/be-careful-inference-2x2-experiments-and-other-cross-cutting-designs}{this
        post and paper} before relying on a factorial design.

    \item See \citet{duflO2003role} for a best practice example of an
        encouragement experimental design that measures spillovers to peers.

    \item \citet{dehejia2005program} for Bayesian experimental analysis

    \item Randomisation procedure: \citet{bruhn2009pursuit}

    \item Multiple hypothesis testing: \citet{list2019multiple}. See also
        \citet{blumenstock2018defaults} Table 3 for example of how to present
        P-values in the context of multiple hypothesis testing, and their
        footnote 25 for additional literature.

    \item Visualisation of study procedure and events:
        \citet{blumenstock2018defaults} Figure 1.

\end{itemize}


\subsubsection*{Quasi-eperimental best practices}

\begin{itemize}
	\item \citet{melnikov2019gangs}: see below, for DiD and RDD
	\item \citet{middleton2018lifting}: for RDD
\end{itemize}

\subsubsection{Misc.}

\begin{itemize}

    \item Use effect of one standard deviation change of x on outcome (e.g.
        \citet{ameriks2003wealth} p. 20).

\end{itemize}


\subsubsection*{\citet{melnikov2019gangs}}

\paragraph{Summary}
Show using a regression discontinuity design that living in a gang-controlled area in EL Salvador has substantially detrimental effects on a number of quality-of-life indicators. They carefully discuss identifying assumptions for the RD design, validate the findings using a difference-in-difference approach with different data, and conduct a battery of robustness tests.

\paragraph{Methods}
The RD design exploits a change in US deportation law in 1997, which led to a large increase in forced removals of gang members from the US back to their native San Salvador. The analysis uses census data from 1992 and 2007 together with detailed maps of gang-controlled areas in El Salvador. The effect of living in a gang-controlled area is then estimated as

\begin{equation*}
	y_i = \alpha_0 + \alpha_1 \text{distance}_i + \alpha_2 \text{gang territory}_i * \text{distance}_i + \alpha_3 \text{gang territory}_i + \epsilon_i,
\end{equation*}

where distance from gang-controlled territory is the forcing variable and $\alpha_3$ the parameter of interest (for more details on RD, see pischke2018regression and lee2010regression). The interaction term captures the difference in slope on distance inside and outside gang-controlled areas. Because the authors use the entire dataset rather than just a small neighbourhood around the cutoff, this is required for unbiased estimates (to see this, visualise fitting a straight line with a discontinuity through all data points -- this would lead to very different estimates of the causal effect). The identifying assumptions are the following: (1) borders or gang-controlled areas need to be as good as randomly assigned, and (2) there must not be selective migration in or out of areas.

In addition, the authors use a difference-in-difference design to estimate the effect of gang presence on economic development (measured by luminosity growth) across all of El Salvador using the following regression model:

\begin{equation}
	\text{luminosity}_{i,t} = \rho_i + \gamma_t + \beta \text{gang presence}_i \times \{\text{Year} > 1997\}+ \epsilon_{i,t}
\end{equation}

where the first two coefficients are area and time fixed effects, and $\beta$ is the causal effect of interest, measuring the average difference in luminosity growth between areas with gang activity and those without.

\paragraph{Key takeaways}
This is a very thorough and convincing application of RD and DiD. Look at this for best practice guidance if I rely on these methods.
