
\section{Transaction data}%
\label{sec:transaction_data}


\citet{farrell2020estimating}
\begin{itemize}
    \item Use gradient boosting machines and features derived from bank
        customer data to estimate customers income and show that their
        predicted income is closer verified incomes than naive approaches like
        summing up identified salary inflows.

    \item They derive features from transaction, account, user, and credit card
        information from JPMorgan Chase customers, and evaluate model
        performance against verified incomes from mortgage applications of a
        subset of these customers.

    \item Paper shows that a relatively straightforward machine learning model
        outperforms naive approaches (like the one I'm using with MDB data). I
        can't use the approach with my MDB data, however, because I don't have
        verified incomes against which to verify my estimates.
\end{itemize}



\citet{anderson2008multiple}
\begin{itemize}
    \item Uses a useful indexing procedure to group mutliple outcomes into
        indices to reduce the problem of multiple hyothesis testing.
\end{itemize}


\citet{viviano2021should}
\begin{itemize}
    \item Provide useful guidelines (in section 4) for how to correct for multiple
        hypothesis testing in applied econometric research.
\end{itemize}
