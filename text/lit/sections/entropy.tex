

\section{Entropy}%
\label{sec:entropy}

\citet{krumme2013predictability}
\begin{itemize}
    \item ...
\end{itemize}

\citet{eagle2010network}
\begin{itemize}
    \item Show that people who have a more diverse set of contacts in terms of
        the number of contacts and the geographic variation of those contacts
        tend to live in more prosperous areas, supporting the idea from
        sociology that a large number of loose contacts is associated with
        prosperity.

    \item They define diversity using entropy, calculating social diversity
        using call volumes of individual uses (p_ij is the share of i's calls
        spend talking to j) and spatial diversity using areas (p_ij is the
        share of i's calls spend talking to contacts in area j).

    \item They also use PCA to create a composite measure of diversity, which
        is even more predictive than either of the two measures combined.
\end{itemize}


\citet{diclemente2018sequences}
\begin{itemize}
    \item Nice visualisation of purchasing patterns in figure 1.
\end{itemize}


\citet{hut2022changes}
\begin{itemize}
    \item First show that neither illnesses associated with poor diet not other
        life circumstances cause changes in diet despite the fact that the
        average diet in the sample is quite poor.

    \item Second part: identify households that do change diet, then ask 1) is
        behaviour change predictable based on life circumstances, baseline
        behaviour, or demographics?, and 2) how do hhs change their behaviour
        (e.g. what food groups are most susceptible to change)?. For 1) they
        use ml (random forest).
\end{itemize}

What I could do:

\begin{itemize}
    \item Create statistics to summarise spending and savings behaviour. Total
        spend for spend, net save for saving. (How consistent are they within
        users?)

    \item Identify users who change (reduce spending by a lot and/or increase
        savings).

    \item Characterise change and try to predict it.

    \item Actually, just focus on savings first, since this is the main policy
        variable.
\end{itemize}

What I need for this to work:
\begin{itemize}
    \item Spend and savings behaviour should be consistent over time for most
        users most of the time.

    \item There should be at least some users for whom we observe changes.

    \item If these two are true, then I can't loose, since I'll be able to
        characterise changes (even if they are different across users) and not
        being able to predict changes is itself informative.
\end{itemize}
