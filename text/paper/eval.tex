\documentclass[a4paper, 11pt]{article}
\usepackage{/Users/fgu/dev/projects/dotfiles/latex/paper}
\bibliography{/Users/fgu/dev/projects/dotfiles/latex/fabib}
\onehalfspacing

\newcommand{\figdir}{../../output/figures}
\newcommand{\tabdir}{../../output/tables}

\title{\textbf{Evaluation\footnote{This research was supported by Economic and Social Research Council grant number ES/V004867/1. WBS ethics code: E-414-01-20.}}}

\author{
    Fabian Gunzinger \\ Warwick Business School
    \and
    Neil Stewart \\ Warwick Business School
}

\date{\today}

\begin{document}

\maketitle
% % !TEX root = ../eval.tex

\begin{abstract}

In this paper, I test whether using Money Dashboard is associated with a
reduction in discretionary spending and an increase in ``rainy-day savings''. I
find that users reduce their discretionary spend by between \pounds100 and
\pounds150 (11-17\% of average discretionary spend) once they start using the
app and sustain that reduction throughout the six-month post-signup period I
consider. Looking at disaggregated measures of discretionary spend further
shows that the reduction is the result of maintained month-to-month changes in
behaviour rather than one-off cancellations of direct-debit transactions, that
it results from reducing spending on a number of different categories of
purchases rather than a single one, and that it is a result of changes along
the extensive rather than the intensive margin -- users reduce the number of
transactions they make rather than the value of the average transaction.
Interestingly, users do not seem to use these additional funds to build up
``rainy-day savings'': net-inflows into savings accounts do not change after
signup. I can also neither find significant increases in flows into investment
and pension accounts or additional savings accounts that are not linked to the
app, not additional loan repayments.

\end{abstract}



\tableofcontents
\newpage

% % !TEX root = ../eval.tex

\section{Introduction}%
\label{sec:introduction}

This paper evaluates whether using Money Dashboard, a UK-based financial
aggregator app, is associated with a reduction in discretionary spend and an
increase in emergency savings -- short-term savings intended as a buffer
against unexpected financial shocks.

The question is important because a large number of adults in the UK and the US
do not have enough savings to cover unexpected expenses like car or medical
bills: in the UK, 25 percent of adults would be unable to cover an unexpected
bill of \pounds300 \citep{phillips2021supporting}, while in the US, about 30
percent would be unable to cover a \$400 bill \citep{fed2022economic}.

A large literature has documented a variety of factors that can influence
undersaving and financial decision-making more generally: time-preferences and
present bias \citep{laibson1997golden, frederick2002time,
read2018intertemporal, ericson2019intertemporal, cohen2020measuring}, inertia
\citep{madrian2001power}, over-extrapolation \citep{choi2009reinforcement},
limited self-control and willpower \citep{thaler1981economic,
  benhabib2005modeling, o1999doing, fudenberg2006dual, loewenstein2004animal,
gul2001temptation}, cognitive limitations and financial literacy
\citep{agarwal2009age, agarwal2013cognitive, korniotis2011older,
agarwal2010learning, fernandes2014financial, jorring2020financial}, attitude
towards money and spending \citep{rick2008tightwads, rick2011fatal}, one's
perceived locus of control \citep{perry2005control}, degree of optimism
\citep{puri2007optimism}, the ability to frame decisions broadly rather than
narrowly \citep{kumar2008decision}, propensity to gamble
\citep{kumar2009gambles}, one's social network \citep{bailey2018economic,
kuchler2021social}, the degree of one's financial planning
\citep{ameriks2003wealth}, and habits \citep{blumenstock2018defaults,
schaner2018persistent, de2013deposit}.\footnote{For two thorough reviews of the
literature, see \citet{agarwal2017shapes} and \citet{greenberg2019financial}.}

Researchers have also identified a number of tools that can help people
overcome some of these biases. Probably the most studied approach is changing
default options. Changes in defaults are consequential and often successful
because people often stick with the status quo \citep{samuelson1988status},
interpret defaults as a recommendation \citep{mckenzie2006recommendations},
implicitly view them as reference points they do not want to move away from
\citep{johnson2003defaults, kahneman1979prospect}, and often end up sticking
with them even if they intend to to otherwise because of procrastination
\citep{carroll2009optimal, ericson2017interaction}. Default options have been
applied across range of areas and have, for instance, been found to increase
retirement savings contributions \citep{thaler2004save, madrian2001power,
beshears2009importance} and organ donations \citep{johnson2003defaults,
gimbel2003presumed, abadie2006impact}.

Another extensively studied approach is the commitment device, whereby an
individual restricts their future choice set in order to avoid choosing a
self-defeating action. Not everyone makes use of these devices when they are
offered \citep{bryan2010commitment} and they do not work in all contexts
\citep{laibson2015don,robinson2018some}, but they have been found to help
individuals increase their savings rates \citep{ashraf2006tying}, quit smoking
\citep{gine2010put}, make healthier food choices \citep{schwartz2014healthier},
and exercise more regularly \citep{royer2015incentives}.

A third approach are implementation intentions, a particular type of planning
for the achievement of one's goals that involves ``if-then'' intentions, such
as ``if I get paid, then I transfer 10 percent of it into my savings account''
\citep{gollwitzer2006implementation, rogers2015beyond}. Such intentions have
been found to support perseverance in pursuing one's goals
\citep{oettingen2010strategies} and to increase overall goal attainment across
different age groups, life domains, and types of obstacles
\citep{gollwitzer2006implementation}.

Finally, social norms messaging, whereby people are informed about how their
own behaviour compares to that of a relevant peer group, has been successful in
inducing high-energy-use households to lower their energy consumption
\citep{schultz2007constructive, allcott2011social, allcott2014short,
brandon2017effects}. Such information can be especially useful for domains
where people usually are uninformed about the behaviour of their peers and do
thus not have a reference point against which to calibrate their own behaviour
-- a situation most people find themselves in when it comes to discretionary
spending and saving. However, social norms messaging can also backfire and has
been found to lower participation in pension savings plans
\citep{beshears2015effect} and completion rates of an online course
\citep{rogers2016discouraged}, making it important to test messages before
deploying them on a large scale to avoid unintended consequences.

FinTech aggregator apps such as Money Dashboard are a promising mean to help
people manage their spending and savings for two reasons: eventually, as these
apps become more sophisticated, they can be used to implement and test the
above mentioned tools at a large scale, and thus potentially bring their
benefit from research labs or specifically designed tools used by a small
minority to the broader population. In the meantime, they provide additional
features that -- on their own -- can help people manage their finances more
successfully. In particular, they provide easy access to financial information
that makes it easier to monitor one's spending and saving, and often also offer
tools such as budgeting and the setting of spending goals. Financial
information that is more easily accessible and is aggregated in ways that help
people keep track of their goals might be beneficial because rational
inattention theory predicts that it makes people more likely to access that
information, which, in turn, might lead to better consumption decisions
\citet{brunnermeier2008wealth, dellavigna2009psychology, sims2003implications}.
Tools that help with budgeting and with setting spending goals have the
potential to help users make consumption decisions more in line with their
intentions because such tools can act as commitment devices that -- if users
experience disutility from falling short of their goals -- introduce a
cognitive cost to overspending or undersaving.

In this paper, I specifically test whether using Money Dashboard -- an
aggregator app that offers aggregation of all financial accounts and budgeting
features -- is associated with a reduction in discretionary spending and an
increase in emergency savings. I use a new estimator proposed by
\citet{callaway2021difference} that corrects for recently identified problems
in two-way fixed effects estimates.

I find that users reduce their discretionary spend by between \pounds100 and
\pounds150 per month (11-17\% of average discretionary spend) once they start
using the app and sustain that reduction throughout the six-month post-signup
period I consider. Looking at disaggregated measures of discretionary spend
further shows that the reduction is the result of maintained month-to-month
changes in behaviour rather than one-off cancellations of direct-debit
transactions, that it results from reducing spending on a number of different
categories of purchases rather than a single one, and that it is a result of
changes along the extensive rather than the intensive margin -- users reduce
the number of transactions they make rather than the value of the average
transaction.  Interestingly, users do not seem to use these additional funds to
build up emergency savings: net-inflows into savings accounts do not change
after signup. I can also neither find significant increases in flows into
investment and pension accounts or additional savings accounts that are not
linked to the app, not additional loan repayments. This suggests that users
either leave unspent funds to accrue in their current accounts (which would not
count to my definition of savings), of allocate them across a number of
different uses in small amounts I am not powered to detect.

There are two main limitations to my approach. First, the data is not
generated by a randomised experiment. The gold-standard to evaluate whether use
of Money Dashboard improves financial outcomes would be a randomised
controlled-trial, where out of a sample of potential users (ideally random and
representative of the UK population), we would randomly grant access to the app
to some users and then compare outcomes of those treated users with the control
group of users who did not have access. Instead, the data I have access to
only contains data for individuals who self-selected into using the app.
Individuals will choose to do so for a number of different reasons, all of
which are unobserved in the data, and at least some of which would probably
have changed their financial outcomes even if they had not signed up to the
app. Any changes in financial outcomes we observe are thus ``aggregate'' or
``net'' effects of these unobservables and the ``pure'' causal effect of app
use.

To see this, think of the net effect as $\textit{net effect} = \textit{causal
effect} + \textit{``need''}(\downarrow) + \textit{``motivation''}(\uparrow)$,
where the arrows indicate the direction of the bias, and consider three cases
that illustrate three stylised but plausible scenarios for signup. First,
consider a user who signs up in the hope that the app will help them reign in
discretionary spending that has has gotten out of hand. If it takes the user
some time to fully adjust their spending, then even if the app does help them
make these adjustments, the estimated positive effect of app use will be biased
downward. Next, consider a user who decides to start bringing their own lunch
to work instead of eating out in an effort to save for a new car and signs up
to MDB in the hope that the app will help them keep track of their spending.
Such a user would probably have reduced their discretionary spend even if they
had not signed up to the app, thus creating an upward bias on our estimated net
effect. Finally, consider a user who signs up to MDB purely because they
happened to see an advert for the app on the Bus and got curious. In this case,
we can think of signup being close to random -- almost as if the user had been
allocated to the treatment group in our ideal experiment -- and the estimated
net effect will closely resemble the causal effect of app use. Hence, under the
weak assumption that at least some users sign up for reasons that are not as
good as random, our estimated effects will be biased upwards or downwards
depending on the relative proportion of users whose unobservable reasons for
signup create an upward and downward bias.

The second limitation is that even if I were able to isolate the effect of
the app, I am not able to differentiate between the contributions of different
features of the app such as improved access to information and budgeting.

Despite these limitations, which mean that we cannot interpret results as
causal effects of using Money Dashboard, the finding that app use is associated
with an economically significant and sustained drop in discretionary spend, and
the finding that this reduction comes about by users making fewer transactions
in a number of different spending categories month-by-month provides
interesting insight of how users reduce discretionary spend, either because of
or simply while using Money Dashboard. Furthermore, the results make it very
plausible that Money Dashboard, and possibly apps like it, does have a
positive causal effect, and thus suggests that further research, focused on
identifying the causal effect of the app overall as well as that of its
component parts, would be interesting and worthwhile.

My work mainly contributes to three strands of the literature. The first is the
nascent literature that studies the effect of FinTech apps on financial
outcomes suggests. This literature suggests that these apps can indeed lead to
improved financial outcomes through a number of different channels: providing
users with information about their spending relative to peers has been found to
reduce spending \citep{dacunto2020crowdsourcing}; offering budgeting options,
to reduce spending \citep{lukas2022influence}; goal setting, to increase
savings \citep{gargano2021goal}; and facilitating access to financial
information, to reduce non-sufficient funds fees \citep{carlin2022mobile} and
discretionary spend \citep{levi2020mind}.

The two studies that are most closely related to my work are
\citet{lukas2022influence} and \citet{gargano2021goal}.
\citet{lukas2022influence} also use data from Money Dashboard to study the
effect of budgeting on discretionary spend. In line with my finding, they find
that budgeting reduces discretionary spend in the associated categories and
that this effect persists throughout the six-month post-budgeting period they
study. Their approach also suffers from the two main limitations of my paper
discussed above: budgeting is not randomly assigned, and they cannot isolate
the effect of budgeting from other MDB features that might help users reduce
their spending. There are two main differences to my work: first,
\citet{lukas2022influence} do not study the effect of MDB use on savings.
Second, they use data from between January 2014 and December 2016, and do not
select their sample to ensure that it only contains users for whom they can
observe all financial accounts. As I discuss in
Section~\ref{sub:preprocessing}, there is a large number of users in the raw
dataset for whom we are unlikely to observe all transactions, making careful
sample selection critical to ensure that observed changes in behaviour do not
merely reflect a shift of transactions between observed and unobserved
accounts. \citet{gargano2021goal} study the effect goal setting on savings and
find that setting savings goals does increase savings rates. Their approach
does not suffer from the two limitations of my paper since they exploit the
random assignment of some users into a group of beta testers that can set
savings goals, and because the savings app that provides the data for their
study was initially designed as a simple savings app and provides no additional
features that could plausibly influence savings behaviour. However, the
flip-side of being able to cleanly identify the effect of goal setting is that
most apps that people use to increase their savings or reduce their spending do
-- like Money Dashboard -- have multiple features that might alter financial
behaviour, so that studying their joint effect is of interest, too.

The second strand of related literature is the very recent literature on studying
interventions to help increase emergency savings an area of household finances
that has until recently had no attention. In addition to studies testing the
effect of FinTech apps on savings, there is a strand of research that studies
the use of auto-enrolment into employer-sponsored savings accounts -- similar
to the ones used to increase pension savings \citep{thaler2004save,
choi2004better, choukhmane2019default} -- and finds an increase in both
participation (relative to opt-in accounts) and account balances
\citep{beshears2020building, berk2022automating}.

Finally, my work also contributes to a rapidly growing literature of using
financial-transaction data from banks or financial aggregator apps to
understand consumer financial behaviour. As already mentioned,
\citet{kuchler2020sticking} use data from a financial aggregator app to
estimate time preferences. Similar data has been used to show that consumer
spending varies across the pay cycle
\citep{gelman2014harnessing,olafsson2018liquid}, to test the consumer spending
response to exogenous shocks \citep{baker2018debt,baugh2014disentangling}, and
to better understand the generational differences in financial platform usage
patterns \citep{carlin2019generational}. Some researchers use transaction-data
directly provided by banks. \citet{ganong2019consumer} show that consumer
spending drops sharply after the predictable income drop from exhausting
unemployment insurance benefits, \citet{meyer2018fully} analyse how individuals
reinvest realised capital gains and losses, and \citet{muggleton2020evidence}
show that chaotic spending behaviour is a harbinger of financial
distress.\footnote{For a comprehensive review of the literature using financial
transaction data, see \citet{baker2022household}.}


The remainder of this paper is organised as follows: Section~\ref{sec:data}
introduces the dataset used, discusses preprocessing and presents summary
statistics; Section~\ref{sec:estimation} introduces the empirical approach used
in the analysis; Section~\ref{sec:results} presents the results; and
Section~\ref{sec:conclusion} concludes. To make it easier for interested
readers to clarify questions about details and subtleties of data preprocessing
and analysis steps, I provide links to the scripts that implement the steps
discussed in the text in the relevant places throughout the text.\footnote{The
    projects GitHub repo that contains all files used to produce the results
    can be found at
\href{https://github.com/fabiangunzinger/mdb\_eval}{https://github.com/fabiangunzinger/mdb\_eval}.}


\section{Introduction}%
\label{sec:introduction}


% % !TEX root = ../eval.tex

\section{Methods}%
\label{sec:data}

\subsection{Data}%
\label{sub:data}

\paragraph{Dataset description:}%
\label{par:dataset_description}

I use data from a UK-based financial management app that allows users to link
accounts from different banks to obtain an integrated view of their finances.
The complete dataset contains more than 500 million transactions made between
2012 and June 2020 by about 250,000 users, and provides information such as
date, amount, and description about the transaction as well as account and
user-level information. Crucially for this paper, the app can access up to
three years of historic data for each linked account.

The main advantage of the data for the study of consumer financial behaviour is
that we can potentially observe user behaviour at the transaction level across
all their accounts, and that the data is collected automatically and in
real-time rather than through surveys that collect data with some time-lag and
often rely on consumer's ability and willingness to provide accurate
information.

There are four main limitations of the data used in this study and data from
financial aggregator apps more genarally\citet{baker2022household}. The first
is that users self-select into using the app, leading to a non-representative
sample of the population. It has been well documented that financial management
app users tend to be male and younger and higher-income earners than the
average citizen \citet{carlin2019generational}. Also, as pointed out in
\citet{gelman2014harnessing}, a willingness to share financial information with
a third party might not only select on demographic characteristics, but also
for an increased need for financial management or a higher degree of financial
sophistication. Because our analysis does not rely on representativeness, we do
not address this.\footnote{For an example of how re-weighing can be used to
mitigate the non-representative issue, see \citet{bourquin2020effects}.} A
second limitation is that while we can observe user's complete financial
behaviour if they add all their financial accounts to the app, it is not
trivial to distinguish between users who do and do not do that. I address this
challenge in the sample selection process documented below. The third issue is
that while the app is able to classify many transactions into types, it
misclassifies some transactions and cannot classify others altogether. I
address this as part of the cleaning process documented below. Finally,
replication of the analysis is not possible because the data is proprietary.


\paragraph{Cleaning:}%
\label{par:cleaning}

I use the dataset described above for a number of projects, and perform a
number of steps to create a minimally cleaned version of the dataset that is
the basis for all such projects. These steps are performed in a dedicated
data repository and not run as part of this project, but the module with
all cleaning functions is available in the project directory.\footnote{Link to
    cleaning functions:
\href{https:/egithub.com/fabiangunzinger/mdb_eval/blob/f51e49c95c5884d2dc417be23921a8acd85aec9d/src/data/clean.py}{\faGithub}}

Here, I briefly describe the main cleaning steps and their rationale. I drop
all transactions with a missing description string because these cannot be
categoriesed, and all transactions that are not automatically categoriesed by
the app. Dropping these transactions makes is likely that we will underestimate
amounts spent and saved, but minimises the risk of incorrectly classified
transactions. I group transactions into transaction, spend, and income
subgroups. Spend subgroups are defined following \citet{muggleton2020evidence};
income subgroups, following \citet{hacioglu2021distributional}.\footnote{Link
to classification file:
\href{https://github.com/fabiangunzinger/mdb_eval/blob/92af366d4c4052cc7a7f78a6178086de8ecdfb75/src/data/txn_classifications.py}{\faGithub}}
Finally, I classify as duplicates and drop transactions with identical user ID,
account ID, date, amount, and transaction description. This will drop some
genuine transactions, such as a user buying two identical cups of coffees at
the same coffee shop on the same day. However, data inspection suggests that in
most cases, we remove genuine duplicates.


\paragraph{Sample selection:}%
\label{par:sample_selection}

We select our sample so as to include users for whom we can be reasonably
certain that we observe all relevant financial transactions, and do so for at
least six months before and after they sign up to the app. In addition to that,
we exclude users who might use the app for business purposes as well as
pensioneers, whose financial objectives might be different.

\begin{table}
\centering
\caption{Sample selection}\label{tab:selection}
\begin{tabular}{lrrrr}
\toprule
                                       &   Users & User-months &        Txns & Txns (m\pounds) \\
\midrule
                            Raw sample & 271,856 &   7,948,520 & 662,112,975 &         124,573 \\
Annual income of at least \pounds5,000 &  92,913 &   2,546,058 & 243,279,442 &          44,616 \\
          At least one savings account &  54,885 &   1,645,698 & 166,547,361 &          32,920 \\
            At least 12 months of data &  47,888 &   1,604,982 & 163,679,001 &          32,436 \\
  At least 6 months of pre-signup data &  30,597 &     848,878 &  88,938,732 &          18,419 \\
  Monthly spend of at least \pounds200 &  16,421 &     451,959 &  52,289,786 &          11,351 \\
           At least 10 txns each month &  15,405 &     420,857 &  49,333,260 &          10,521 \\
      Complete demographic information &  12,318 &     347,132 &  40,993,232 &           8,541 \\
                           Working age &  12,085 &     339,487 &  40,260,306 &           8,232 \\
                          Final sample &  12,085 &     339,487 &  40,260,306 &           8,232 \\
\bottomrule
\end{tabular}

\tabnote{\textwidth}{Number of users, user-months, transactions, and
transaction volume in millions of British Pounds left in our sample after each sample selection step. Link to sample selection
code:
\href{https://github.com/fabiangunzinger/mdb_eval/blob/main/src/data/selectors.py}{\faGithub}.}
\end{table}


Table~\ref{tab:selection} lists the precise conditions we applied to implement
these criteria and their effect on sample size. We remove the first and last
month of data for all users because we are unlikely to observe all transactions
for these months. We also drop test users, since their objectives for app use
might have been different from ordinary users.\footnote{We cannot identify test
users precisely, but drop users who signed up prior to or during the first year
the app was in operation.}

To ensure that we observe users for at least 12 months around app signup, we
require 6 months of data before the signup month, and another five months after
the signup month. Our main outcome variable is netflows into a user's savings
accounts. It is thus critical that we observe enough historical data for these
savings accounts to ensure that we observe all transactions during our 12 month
perdiod of interest. This is complicated by the fact that we cannot see when an
account was opened at the bank, but only when it was added to the app. While
cases where a user adds an account to the app as soon as it was opened are
unproblematic, users will often add accounts after they were opened, either
because they have accounts that they opened before signing up to the app, or
because they opened new accounts after signup but add them to the app with a
delay. In such cases, it is critical that, once the account is added, we
observe the complete historical data up to 6 months before signup or up to the
month in which the account was opened, whichever happened later. To see why
this is critical, imagine a scenario where a user opens an account 10 months
before they sign up to the app, makes a monthly transfter to the account of
\pounds100, adds the account to the app on signup, but we onserve only 3 months
of historical data. In this case we would observe that the user saved
\pounds300 before signup and \pounds600 after, and erroneously conclude that
post signup savings were twice as high. The most extreme case we need to cover
is that of a user opening a savings account more than six months before signup
and adding the account to the app five months after signup, in which case we
need to be sure to observe 12 months of historical data. As shown in
Appendix~\ref{app:data}, all major banks started providing 12 months of
historical data for current and savings accounts from April 2017 onwards, which
is why we restrict our sample to users who signed up in or after that month.

To ensure that we can be reasonably certain to observe users have added all
their financial accounts to the app, we restrict our sample to users with at
least one savings and current account, with an annual income of at least
\pounds5,000, and a minimum of 10 transactions and a spend of \pounds200 every
month. To remove users who might use the app for business purposes, we drop
users with more than 10 active accounts in any given month. Finally, we remove
users for whom we cannot observe all demographic information we use as
covariates in our analysis, and users who are not between the ages of 18 and
65, as their financial objectives are plausibly different.


\paragraph{Data transformations:}%
\label{par:data_transformations}

To minimise the influence of outliers, we winsorise spend, income, and savings
accounts flow variables at the 1 percent level or -- if we winsorise on both
ends of the distribution -- at the 0.5 percent level.


\paragraph{Summary statistics:}%
\label{par:summary_statistics}

Figure~\ref{fig:sample_description} describes the sample.

\begin{figure}[H]
    \centering
    \caption{Sample characteristics}
    \includegraphics[width=\linewidth]{\figdir/sample_description.png}
    \label{fig:sample_description}
    \fignote{\textwidth}{Panels A and B show the distribution of disposable
        income and total spending in 2019, respectively, benchmarked against
        the 2018/19 wave of the ONS Living Cost and Food Survey (LCFS). The
        remaining panels show the data distributions of age, gender, region,
    and the number of active accounts.}%
\end{figure}

Table~\ref{tab:sumstats} provides summary statistics.


% Table created by stargazer v.5.2.3 by Marek Hlavac, Social Policy Institute. E-mail: marek.hlavac at gmail.com
% Date and time: Fri, Jul 08, 2022 - 16:36:19
\begin{table}[H] \centering 
  \caption{Summary statistics} 
  \label{tab:sumstats} 
\footnotesize 
\begin{tabular}{@{\extracolsep{5pt}}lccccccc} 
\\[-1.8ex]\hline 
\hline \\[-1.8ex] 
Statistic & \multicolumn{1}{c}{Mean} & \multicolumn{1}{c}{St. Dev.} & \multicolumn{1}{c}{Min} & \multicolumn{1}{c}{Pctl(25)} & \multicolumn{1}{c}{Median} & \multicolumn{1}{c}{Pctl(75)} & \multicolumn{1}{c}{Max} \\ 
\hline \\[-1.8ex] 
user\_id & 503,356.1 & 34,778.6 & 441,309 & 474,770 & 499,754 & 528,755 & 575,009 \\ 
ym & 587.2 & 9.9 & 560 & 580 & 588 & 595 & 606 \\ 
ymn & 201,854.3 & 86.1 & 201,609 & 201,805 & 201,901 & 201,908 & 202,007 \\ 
month & 6.5 & 3.5 & 1 & 3 & 6 & 9 & 12 \\ 
txns\_count & 113.5 & 60.4 & 10 & 71 & 102 & 143 & 335 \\ 
txns\_volume & 18,110.9 & 25,559.3 & 227.1 & 6,036.4 & 10,291.1 & 18,875.1 & 180,795.9 \\ 
month\_income & 2,931.6 & 2,238.4 & 417.3 & 1,510.9 & 2,243.2 & 3,567.6 & 13,016.6 \\ 
inflows & 748.7 & 2,498.0 & 0.0 & 0.0 & 0.0 & 400.0 & 19,353.4 \\ 
outflows & 738.6 & 2,388.1 & 0.0 & 0.0 & 0.0 & 370.0 & 17,933.6 \\ 
netflows & 15.2 & 2,894.2 & $-$19,940.9 & 0.0 & 0.0 & 50.0 & 22,000.0 \\ 
netflows\_norm & $-$0.001 & 1.0 & $-$7.5 & 0.0 & 0.0 & 0.02 & 7.7 \\ 
inflows\_norm & 0.3 & 0.9 & 0.0 & 0.0 & 0.0 & 0.2 & 6.7 \\ 
outflows\_norm & 0.3 & 0.9 & 0.0 & 0.0 & 0.0 & 0.2 & 6.7 \\ 
has\_pos\_netflows & 0.3 & 0.5 & 0 & 0 & 0 & 1 & 1 \\ 
pos\_netflows & 413.9 & 2,078.5 & 0.0 & 0.0 & 0.0 & 50.0 & 22,000.0 \\ 
user\_reg\_ym & 590.8 & 5.4 & 580 & 587 & 590 & 595 & 601 \\ 
t & 0.4 & 0.5 & 0 & 0 & 0 & 1 & 1 \\ 
tt & $-$3.6 & 11.2 & $-$35 & $-$10 & $-$3 & 4 & 26 \\ 
month\_spend & 2,742.6 & 2,613.1 & 200.1 & 1,198.0 & 1,974.8 & 3,288.0 & 16,936.3 \\ 
age & 36.7 & 10.1 & 18 & 29 & 35 & 43 & 65 \\ 
is\_female & 0.4 & 0.5 & 0 & 0 & 0 & 1 & 1 \\ 
is\_urban & 0.8 & 0.4 & 0 & 1 & 1 & 1 & 1 \\ 
region\_code & 4.0 & 3.2 & 0 & 1 & 4 & 6 & 13 \\ 
has\_savings\_account & 1.0 & 0.0 & 1 & 1 & 1 & 1 & 1 \\ 
has\_current\_account & 1.0 & 0.0 & 1 & 1 & 1 & 1 & 1 \\ 
generation\_code & 2.6 & 0.7 & 1 & 2 & 3 & 3 & 4 \\ 
prop\_credit & 0.1 & 0.2 & 0.0 & 0.0 & 0.0 & 0.1 & 1.0 \\ 
dspend & 859.4 & 748.0 & 0.0 & 365.6 & 653.6 & 1,109.4 & 4,236.4 \\ 
dspend\_count & 39.8 & 27.9 & 0 & 21 & 35 & 52 & 465 \\ 
dspend\_mean & 24.7 & 42.1 & 0.2 & 13.8 & 19.5 & 27.6 & 7,508.2 \\ 
accounts\_active & 3.3 & 1.8 & 1 & 2 & 3 & 4 & 10 \\ 
accounts\_total & 5.8 & 2.8 & 2 & 4 & 5 & 8 & 19 \\ 
\hline \\[-1.8ex] 
\end{tabular} 
\end{table} 


We use data from the 2018-2019 wave of the Office of National Statistics' Living Costs and Food
Survey (LCFS).\footnote{We accessed the data via the UK Data Service at the
following url:
\url{https://beta.ukdataservice.ac.uk/datacatalogue/studies/study?id=8686}.}
Data covers the period between April 2018 and March 2019.


\subsection{Estimation}%
\label{sub:estimation}

We want to estimate the effect of app use over time.  Given our data, a natural
way to do this would be to use a dynamic two-way fixed effects model that
includes user and year-month fixed effects and dummies indicating time since
app signup. The estimated coefficients on these dummies are then conventionally
interpreted as dynamic treatment effects. However, a series of recent papers
have documented that while such an approach is frequently used in applied
research, the parameter estimates from dynamic two-way fixed effects models are
not valid estimators of dynamic treatment effect in most settings. In
particular, in settings with staggered treatment assignment, where units are
first exposed to treatment at different points in time (as is the case in our
setting), dynamic two-way fixed effects are valid only if there is homogeneity
in treatment effects across treatment adoption cohorts. In most settings,
including our own, this is a very strong assumption.\footnote{Two excellent
reviews of this new literature are \citet{roth2022trending, baker2022much}.}

Because of this, I use a new estimator proposed by
\citet{callaway2021difference}, which allows for arbitrary treatment effect
heterogeneity across treatment adoption cohorts and time, and allows for the
incorporation of a parallel trends assumption conditional on covariates. In
describing the estimator, I follow the approach of
\citet{callaway2021difference} of first defining the causal parameter of
interest, and then discussing identification, estimation, and inference.

The basic building block of the framework, and the causal effect of interest,
is the group-time average treatment effect: the average treatment effect at
time $t$ for the group of individuals first treated at time $g$, defined as:

\begin{equation}
    ATT(g,t) = \mathop{\mathbb{E}}[Y_{i,t}(g) - Y_{i,t}(0)|G_i =
    g],
\end{equation}

where $Y_{i,t}(g)$ is the potential outcome in time period $t$ of an individual
$i$ in group $g$, and $Y_{i,t}(0)$ is the (counterfactual) potential outcome of that same
individual if they had remained untreated.

These effects are identified if two main assumptions hold: if there is limited
and known anticipation of treatment, and if the assumption of parallel trends
between treatment and comparison groups holds either unconditionally or
conditionally on a set of covariates.\footnote{Additional assumptions are (i)
    that the treatment is absorbing in the sense that once an individual is
    treated they will remain treated forever, (ii) that individuals in the data
    are randomly and independently drawn from a larger population, and (iii) an
    overlap condition that ensures that there is a positive number of users
    that is first exposed to the treatment at any period and that -- under the
    conditional parallel trends assumption -- propensity scores for initial
    treatment times based on covariates are bounded away from zero. The first
    assumption could be violated only if a user closes their account on the app
    and then signed up again later on, all within the roughly within the
    two-year data periods I use. This cannot be more than a tiny minority of
    users.  The second assumption holds less trivially. One way to think of a
    super population from which the users in the dataset are drawn is to think
    of knowledge about the app as partially random, and about the super
    population of all individuals who would have signed up had they learned
    about the apps existence.} Because the purpose of this section is to convey
    the core idea of the estimation approach, I keep things as simple as
    possible and discuss only the case with no anticipation effects and where
    the parallel trend assumption holds unconditionally.
    \citet{callaway2021difference} show that the same overall approach also
    works when allowing for known anticipation and conditional parallel
    trends.\footnote{Intuitively, known anticipation merely shifts the
        reference period from the period immediately before treatment to the
        period before anticipation of treatment begins. When relying on the
        conditional parallel trends assumption, the overall the group-time
        average treatment effect $ATT(g, t)$ is the average of unconditional
        group-time average treatment effects for each value of the covariate
        vector $X_i$.  As discussed in \citet{roth2022trending}, estimation is
        challenging when $X_i$ is continuous or can take on a large number of
        values, since then we will typically lack data to estimate
        unconditional group-time treatment effects for each value of $X_i$.
        There are different semi- and non-parametric approaches that can be
        used in such cases. Below, I use the doubly-robust estimator, which is
    the default in the `did` package. See \citet{callaway2021difference,
roth2022trending} for more details.}

Given these two assumptions, \citet{callaway2021difference} show that
$ATT(g,t)$ is identified by comparing the expected change of group $g$ between
periods $t$ and $g-1$ with that of a comparison group that is not yet treated
at time $t$:

\begin{equation}
    ATT(g,t) = \mathop{\mathbb{E}}[Y_{i,t} - Y_{i,g-1}|G_i = g] -
    \mathop{\mathbb{E}}[Y_{i,t} - Y_{i,g-1}|G_i = g'], \text{ for any $g'$ > t} 
\end{equation}

As this holds for all $g'$ that are not yet treated at time $t$, it also holds
for an average over all groups in the set $\mathcal{G}$ containing all groups
$g'$ for which $g' > t$,

\begin{equation}
    ATT(g,t) = \mathop{\mathbb{E}}[Y_{i,t} - Y_{i,g-1}|G_i = g] -
    \mathop{\mathbb{E}}[Y_{i,t} - Y_{i,g-1}|G_i \in \mathcal{G}].
\end{equation}

This equation encapsulates two main results: that the period just before
treatment, $g-1$, is a valid reference period, and that the group of all
individuals that have not yet been treated at time $t$ are a valid comparison
group for estimating treatment effects in time $t$.\footnote{If a group of
never-treated individuals is available, then these could also serve as a
comparison group. But because my dataset does not contain such individuals, I
do not discuss this case.} As a result of this, units who are treated in the
very first period in the dataset are dropped from the sample, since there
exists no possible control group based on which to identify their treatment
effect, and since they are not useful as a control group themselves. Similarly,
unit treated in the very last period in the data are also dropped, since there
exists no ``not-yet-treated`` group that could serve as a comparison group for
them.

$ATT(g,t)$ can then be estimated by replacing expectations with their sample
analogues,

\begin{equation}
    \widehat{ATT(g,t)} = \frac{1}{N_g}\sum_{i:G_i=g}[Y_{i,t} - Y_{i, g-1}] -
    \frac{1}{N_\mathcal{G}}\sum_{i:G_i \in \mathcal{G}}[Y_{i,t} - Y_{i, g-1}]
\end{equation}

Once these building blocks are estimated, aggregating them to event-study type treatment effects that provide the (weighted)
average treatment effect $l$ periods away from treatment adoption across
different adoption groups can be achieved by simply calculating

\begin{equation}
    ATT_l = \sum_g w_g ATT(g,g+l),
\end{equation}

where the group weights $w_g$ are the groups relative frequencies in the
treated population. When calculating these event study parameters, I use a
panel balanced in event times, with all units being observed for at least 5
treatment periods. This avoids the $ATT_l$ being influenced by different
group compositions at different periods $l$.\footnote{See Section 3.1.1 in
\citet{callaway2021difference} for a more detailed discussion.}


\subsection{Variables}%
\label{sub:variables}



\paragraph{Treatment}%
\label{par:treatment}

A user changes treatment status from untreated to treated when they start using
the app. Figure~\ref{fig:treatplot_sample_raw} shows the treatment history for
200 randomly selected users.

\begin{figure}[H]
    \centering
    \caption{Treatment assignment plot}%
    \includegraphics[width=0.8\linewidth]{\figdir/treatplot_sample_raw.png}
    \label{fig:treatplot_sample_raw}

    \fignote{\textwidth}{Each horizontal line shows the observed pre and post
        signup periods in blue and red, respectively, for one of 200 randomly
        selected users. The faint vertical white lines indicate month borders,
        whitespace indicates periods in which we do not observe the user. To
    the left of the observed period, this is because the app cannot access data
before that point when the user signs up; to the right, because they have
stopped using the app.}

\end{figure}


\paragraph{Outcomes}%
\label{par:outcomes}

Savings... see Table~\ref{tab:vars} for details.

For a more nuanced understanding of how app use affects savings we also
consider net-savings -- total savings account inflows minus outflows -- as a
proportion of monthly income to see whether a willingness to save more might be
offset by a (later) need to withdraw funds, and a dummy variable for whether a
user has any savings account inflows in a given month to see whether the app
helps users save at all. To investigate possible channels, we consider total
spend, highly discretionary spend, banking charges, the total amount of
borrowing, as well as payday borrowing, all as proportion of monthly income.



Net savings (\textit{netflows\_norm})
Inflows into minus outflows out of all of a user's savings accounts divided
by monthly income. To capture only ``user-generated'' flows, we exclude
interest and ``save the change'' transactions, as well as transactions of
less than \pounds5 in absolute
value. Monthly income and raw inflows and outflows are winsorised at the 1
percent level.
We focus on net inflows to capture effective savings.

Positive net savings dummy (\textit{has\_pos\_netflows})
Dummy equal to 1 if there were positive net savings (as defined above).
Captures extensive margin of savings (change in number of months with positive
net deposits)

Positive net savings (\textit{pos\_netflows})
Equal to net savings if there were positive net savings.
Captures intensive margin of savings (change in deposit amount in months with
positive net deposits)

\begin{figure}[H]
    \centering
    \caption{Savings patterns}%
    \includegraphics[width=\linewidth]{\figdir/savings_patterns.png}
    \label{fig:\figdir/savings_patterns}
    \fignote{\textwidth}{Panel A shows distribution of savings account inflow
        amounts, making clear that most transactions are the kinds of round
        amounts we would expect savings transactions to be. The data is
        truncated at \pounds1000. Panel B hows inflows, outflows, and netflows
        into savings accounts for six months before and five months after app
    use.}

\end{figure}


% \paragraph{Adjusting for multiple hypothesis tests}%
% \label{par:adjusting_for_multiple_hypothesis_tests}
% We think of our secondary outcomes as exploratory and do not make any
% adjustments for multiple hypothesis testing.\footnote{For a recent
% game-theoretically motivated discussion of when and how to correct for multiple
% hypothesis testing, see \citet{viviano2021should}.} An alternative approach,
% based on \citet{anderson2008multiple}, would be to group outcomes into
% ``savings'', ``spending'', ``borrowing'', and ``fees'', and consider them as
% different dimensions of a latent variable of interest which we might call
% ``financial management skills''. We do not do that for two reasons: first and
% foremost, because we think it is natural to think of the amount saved as the
% ultimate outcome and of other outcomes as providing a more nuanced
% understanding of savings behaviour or as suggesting possible channels through
% which app use affects savings. Thinking of savings as the main goal is also
% reflected in Money Dashboard's main promise, which is to help users spend less
% and save more, as shown in Figure~\ref{fig:mdb_website}. Second, as pointed out
% in \citet{carlin2017fintech}, incurring overdraft fees is not an unambiguous
% sign of a financial mistake, as the opportunity to go into overdraft confers a
% benefit to the consumer.\footnote{For further discussions on fees, see
% \citet{jorring2020financial, stango2009consumers}.}


\paragraph{Covariates}%
\label{par:covariates}

We control for baseline behaviour, events, and personal characteristics that,
to various degrees, capture a person's need, capacity, motivation, and
awareness to save. Table~\ref{tab:vars} lists all covariates used
together with their definition and the rationale for including them. For all
variables, we include contemporaneous values as well as lags for up to 6
periods. In addition, we control for the previous six months of savings to
capture time-invariant unobserved drivers of savings behaviour (in
specifications without fixed effects) as well as a possible signal for a higher
or lower need for future savings.

Following \citet{vanderweele2019principles} we include covariates that affect
either outcomes or the propensity for treatment or both, exclude from this
set of variables those that are instruments (affect the outcome only through their effect on
treatment propensity) and add to it proxies for unobserved variables that are a
common cause of both outcomes and treatment propensity.\footnote{
\citet{vanderweele2019principles} calls this the ``modified disjunctive cause
criterion'' for covariate selection, as it includes the set of variables that are causally
related to either outcomes, or treatment propensity, or both, but modified to
account for potential bias by excluding instruments and including proxies of
unobserved causes of both outcomes and treatment.}

The table below describes the construction and rationale for including of all
variables used. The code used to construct the variables is available on
\href{https://github.com/fabiangunzinger/mdb_eval/blob/d094f8cd364f64bbe3d4e644abbff726af86de2f/src/data/aggregators.py}{GitHub}.

\begin{table}[htpb]
    \centering\scriptsize
    \caption{Covariates}
    \label{tab:vars}
    \begin{tabularx}{\textwidth}{>{\raggedright\arraybackslash}X
        >{\raggedright\arraybackslash}X>{\raggedright\arraybackslash}X}
    \hline\hline
    Variable (name in dataset) & Definition  & Rationale \\
    \hline\\
    \multicolumn{3}{c}{\textbf{Primary outcome}}\\\\

    \multicolumn{3}{c}{\textbf{Covariates}}\\\\

    New loan dummy (\textit{new\_loan})&
    Dummy variable equal to 1 if user takes out a new loan. Calculated positive inflows
    of funds tagged as ``loan''.&
    Might increase (additional funds) or decrease (need to repay) propensity to
    save in month of takeout and lower propensity to save in the future due to
    need to repay.\\

    Unemployment benefits dummy (\textit{unemp\_benefits})&
    Dummy variable equal to 1 if user has inflow of funds tagged as ``job
    seeker benefits''.&
    Might lower a user's ability to save but increase their need for a money
    management app.\\

    Monthly income (\textit{month\_income})&
    Average monthly income in a calendar year, calculated as the sum of all
    credits tagged income payments in said year divided by 12.&
    Income may alter the need and ability to save and correlate with cognitive
    characteristics that alter a person's propensity to use a money management
    app.\\

    \hline\hline
    \end{tabularx}
\end{table}




\subsection{Code access}%
\label{sub:code_access}

We provide links to code that creates key elements of the paper such as
variable definitions and sample selection directly in the relevant places in
the paper so they can be accessed conveniently. The links are indicated with
the GitHub logo, \faGithub. The hope is that this helps the
curious reader clarify questions about subtleties they might have while reading
the paper. The complete projects GitHub repo is at
\href{https://github.com/fabiangunzinger/mdb\_eval}{https://github.com/fabiangunzinger/mdb\_eval}.



\section{Methods}%
\label{sec:data}

While we were unable to pre-register the analysis because we have had access to
and been working with the Money Dashoard data for months, we proceeded in the
same spirit: we first wrote a draft of the paper in the form of a pre-analysis
plan, following \citet{olken2015promises}, then tested the entire code base --
data pre-processing, balance checks, main analysis, and extensions -- with a 1
percent sample, and finally ran the entire analysis.


\subsection{Dataset}%
\label{sub:dataset}


\subsection{Sample selection}%
\label{sub:sample_selection}

To assess the impact of MDB use on users' financial behaviour we need to
observe their relevant financial history for a sufficiently long period of time
prior to and after they started using the app. For our purpose here, ``relevant
financial history'' includes the complete set of spending transactions and all
savings account inflows and outflows, and ``sufficiently long period'' is a
period of 6 months prior to and after signup, with the month of signup being
the first month of the latter period.\footnote{In
Appendix~\ref{sub:alternative_matching_method} we show results with different
window lengths. \edit{The results are unchanged.}}

Table X provides an overview of the precise conditions we applied to implement
these criteria and their effect on the sample size. The set of functions that
implement each condition can be found on \href{path-to-github}{path to github}.


\subsection{Treatment}%
\label{sub:treatment}

Provide information about signup patterns


\subsection{Outcomes}%
\label{sub:outcomes}

\paragraph{Primary outcome}%
\label{par:primary_outcome}

Our main outcome variable is savings as a proportion of monthly income, where
we measure savings as the sum of all savings account inflows.


\paragraph{Secondary outcomes}%
\label{par:secondary_outcomes}

For a more nuanced understanding of how app use affects savings we also
consider net-savings -- total savings account inflows minus outflows -- as a
proportion of monthly income to see whether a willingness to save more might be
offset by a (later) need to withdraw funds, and a dummy variable for whether a
user has any savings account inflows in a given month to see whether the app
helps users save at all. To investigate possible channels, we consider total
spend, highly discretionary spend, banking charges, the total amount of
borrowing, as well as payday borrowing, all as proportion of monthly income. We
think of these additional outcomes as exploratory and do not make any
adjustments for multiple hypothesis testing.\footnote{For a recent
game-theoretically motivated discussion of when and how to correct for multiple
hypothesis testing, see \citet{viviano2021should}.}

An alternative approach, based on \citet{anderson2008multiple}, would be to
group outcomes into ``savings'', ``spending'', ``borrowing'', and ``fees'', and
consider them as different dimensions of a latent variable of interest which we
might call ``financial management skills''. We do not do that for two reasons:
first and foremost, because we think it is natural to think of the amount saved
as the ultimate outcome and of other outcomes as providing a more nuanced
understanding of savings behaviour or as suggesting possible channels through
which app use affects savings. Thinking of savings as the main goal is also
reflected in Money Dashboard's main promise, which is to help users spend less
and save more, as shown in Figure~\ref{fig:mdb_website}. Second, as pointed out
in \citet{carlin2017fintech}, incurring overdraft fees is not an unambiguous
sign of a financial mistake, as the opportunity to go into overdraft confers a
benefit to the consumer.\footnote{For further discussions on fees, see
\citet{jorring2020financial, stango2009consumers}.}

\subsection{Covariates}%
\label{sub:covariates}

Description of covariates.


\subsection{Difference-in-difference}%
\label{sub:difference_in_difference}

Control group design:
\begin{itemize}

    \item We only have data for a self-selected group of people who choose to
        use Money Dashboard. By virtue of signing up to an app that helps them
        manage their money, these users are different from those who don't sign
        up. As a result, we are unable to answer the question of whether use of
        Money Dashboard helps the average person in the population as a whole
        save more.\footnote{One way to get closer to that answer is to
            re-weight our sample on observable demographic variables so as to
            match the UK population as a whole. But our sample differs from the
            population as a whole both is ways that are observable (demographic
            variables) and unobservable (self-awareness that they need help
            managing their money, cognitive resources to engage with the app,
            motivation to do so). Re-weighting would only help us deal with the
        first of these.} Instead, we are answering the question whether Money
        Dashboard succeeds in helping its \textit{users} save more.

    \item Money Dashboard can access up to three years of historic data for
        each account a user links to their account.

    \item Each user for whom we have sufficient data thus serves as both a
        treatment unit and a potential control unit.

    \item We use a difference-in-differences design to estimate the effect of
        app use. Because we do not have a separate control group, we use the
        per-signup data of Money Dashboard users as control periods and use
        matching to find comparable control user for each tretment user.

    \item To perform the matching, we proceed as follows:

    \item Selection of covariates: all variables that simultaneously affect
        treatment and outcomes. No need to control for fixed effects: these
        capture unobserved time-invariant factors that make an individual sign
        up to MDB and affect its spending habits. Given that these are time
        invariant, and that all users eventually sign up, there is no
        difference between control and treatment units in these factors. Month
        of year: should probably include, as can affect p of signup and
        spending behaviour.

    \item For treatment units, we select data for six months before and after
        signup, where the month of the signup is treated as the first of the
        six after-signup months. For each user, we then calculate the mean
        value of each covariate for the pre-signup period.

    \item We construct potential control units as all 12-month data windows we
        observe before signup, and for each potential control unit calculate
        the mean value of each covariate for the first six months.


    \item Calculate propensity scores and eliminate obs outside common support?


    \item We use matching proceedure introduced in \citet{ho2007matching} and
        implemented in the \textit{MatchIt} R-package \citep{stuart2011matchit} to
        find most similar comparison unit for each treatment unit.

    \item Choice of exact matching proceedure:

        \begin{itemize}

            \item \edit{Matching with or without replacement? I'd think with,
                but read papers and check vignettes for trade-offs.}
                \citet{ho2007matching}: if many more good (with common support)
                control than treatment units available, use many to one
                matching, else do matching with replacement. Check for common
                support using convex hull analysis from
                \citet{king2006dangers}.

            \item Match more exactly on variables that are more correlated with
                outcome. How to determine? Correlation? Cutoff for "high"
                correlation?

            \item We start with exact matching (matching all possible control
                units that exactly match the treatment unit). For continuous
                variables, we use 6 buckets (default in
                \citet{stuart2011matchit}).

            \item If we match more than X percent of treatment observations, we
                stop. If not, we move to another approach. The goal of
                proceedure is to avoid biased estimates due to the deletion of
                too many treatment units \citep{rosenbaum1985bias}.

            \item We use nearest-neighbour matching based on propensity score.
                If we succeed in balancing covariates, we stop. If we don't we
                use more sophisticated specifications to estimate the
                propensity score. See \citet{ho2007matching} section 6.4.

            \item We asses balance following
                \href{https://kosukeimai.github.io/MatchIt/articles/assessing-balance.html}{vignette}.

            \item Check for reduction in moden dependence producing equivalent
                of Fig. 2 in \citet{ho2007matching}.

        \end{itemize}
\end{itemize}


Open questions:
\begin{itemize}
    \item Do we include fixed effects after matching? Reason not to: we use the
        same units for treatment and control, so time-invariant unobserved
        differences should be equally distributed across treatment and control.

    \item How does event studies in \citet{sun2021estimating} relate to all
        this? Is key difference that event studies use periods since treatment
        rather than time?

\end{itemize}


% Estimation equation for static comparison:
% \begin{equation}
%     y_{it} = \alpha + 
% \end{equation}

Estimating treatment effects
\begin{itemize}

    \item Our estimate is the ATT, not the ATE. 

    \item First, we present pre and post signup comparisons without matching
        (i.e. control group is user pre-signup).

    \item Second, we present (static) pre-post using matched comparisons.

    \item Third, we present dynamic pre-post using matched comparisons. Need to
        think about how this relates to \citet{sun2021estimating}, who propose
        an unbiased estimator for dynamic two-way FE event study designs. I
        think our analysis nests theirs, since we might still want to include
        fixed and time effects, though I need to think about this. (Reason to
        do so: we won't be able to match perfectly, so including user and time
        fixed effects to control for unit and time invariant variation still
        seems useful).

    \item As alternative to matching, consider synthetic controls for
        disaggregated data \citep{abadie2021penalized}.

\end{itemize}

Is estimate causal?
\begin{itemize}
    \item \citet{king2006dangers} show that there are four sources of bias
        (ommitted variable, posttreatment, interpolation, extrapolation).
    
    \item Discuss each in turn to argue that effect is causal (for our population
        of interest, which are people signing up to MDB). 
\end{itemize}


% % !TEX root = ../eval.tex

\section{Results}%
\label{sec:results}

\subsection{Static TWFE}%
\label{sub:static_results}

\begin{equation}
    y_{it} = \alpha_i + \lambda_t + \beta D_{it} + \gamma X_{it} + \epsilon_{it}
\end{equation}

Notes:
\begin{itemize}
    \item Assumption: there are no confounding effects (either time-varying,
        individual varying, or individual-time varying), so treatment
        assignment is as good as random.

    \item With controls, we assume that there are no confounding variables
        other than the ones we control for.
\end{itemize}

\begin{table}[htbp]
   \centering
   \tiny
   \begin{threeparttable}[b]
      \caption{\label{tab:reg_static} Static results}
      \begin{tabular}{lcccccccc}
         \tabularnewline \midrule \midrule
         Dependent Variables: & \multicolumn{4}{c}{Net-inflows} & \multicolumn{4}{c}{Discretionary spend}\\
         Model:          & (1)             & (2)             & (3)            & (4)             & (5)              & (6)             & (7)            & (8)\\  
         \midrule
         \emph{Variables}\\
         App use         & 49.96$^{***}$   & 14.21           & 36.60$^{**}$   & 0.22            & 71.37$^{***}$    & 0.84            & 53.08$^{***}$  & -22.61$^{***}$\\   
                         & [19.11; 80.82]  & [-31.50; 59.92] & [6.08; 67.12]  & [-49.16; 49.61] & [65.39; 77.36]   & [-13.45; 15.13] & [43.96; 62.19] & [-33.62; -11.61]\\   
         Month income    & 0.07$^{***}$    & 0.07$^{***}$    & 0.08$^{***}$   & 0.08$^{***}$    & 0.02$^{***}$     & 0.02$^{***}$    & 0.01$^{***}$   & 0.01$^{***}$\\   
                         & [0.06; 0.08]    & [0.05; 0.09]    & [0.05; 0.10]   & [0.05; 0.11]    & [0.02; 0.02]     & [0.02; 0.03]    & [0.00; 0.02]   & [0.00; 0.01]\\   
         Month spend     & -0.12$^{***}$   & -0.12$^{***}$   & -0.16$^{***}$  & -0.16$^{***}$   & 0.16$^{***}$     & 0.16$^{***}$    & 0.12$^{***}$   & 0.12$^{***}$\\   
                         & [-0.13; -0.11]  & [-0.14; -0.10]  & [-0.18; -0.14] & [-0.19; -0.13]  & [0.16; 0.16]     & [0.15; 0.17]    & [0.11; 0.12]   & [0.11; 0.12]\\   
         Active accounts & 38.80$^{***}$   & 37.49$^{***}$   & 70.26$^{***}$  & 68.78$^{***}$   & 46.84$^{***}$    & 42.56$^{***}$   & 86.86$^{***}$  & 75.18$^{***}$\\   
                         & [30.04; 47.56]  & [23.46; 51.53]  & [53.08; 87.44] & [46.71; 90.85]  & [45.14; 48.54]   & [39.57; 45.55]  & [81.78; 91.95] & [70.63; 79.72]\\   
         Intercept       & -6.05           &                 &                &                 & 171.35$^{***}$   &                 &                &   \\   
                         & [-41.06; 28.95] &                 &                &                 & [164.56; 178.14] &                 &                &   \\   
         \midrule
         \emph{Fixed-effects}\\
         Year-month      &                 & Yes             &                & Yes             &                  & Yes             &                & Yes\\  
         User ID         &                 &                 & Yes            & Yes             &                  &                 & Yes            & Yes\\  
         \midrule
         \emph{Fit statistics}\\
         Observations    & 148,932         & 148,932         & 148,932        & 148,932         & 148,932          & 148,932         & 148,932        & 148,932\\  
         R$^2$           & 0.00879         & 0.00932         & 0.03981        & 0.04033         & 0.41708          & 0.42534         & 0.66066        & 0.66973\\  
         Within R$^2$    &                 & 0.00869         & 0.00933        & 0.00922         &                  & 0.41021         & 0.28163        & 0.23660\\  
         \midrule \midrule
         \multicolumn{9}{l}{\emph{Signif. Codes: ***: 0.01, **: 0.05, *: 0.1}}\\
      \end{tabular}
   \end{threeparttable}
\end{table}





\subsection{Dynamic TWFE}%
\label{sub:dynamic_results}

\begin{equation}
    y_{it} = \alpha_i + \lambda_t + \sum^{5}_{\substack{s=-6 \\
    s\neq-1}} \beta_s D_{its} + \gamma X_{it} + \epsilon_{it},
\end{equation}

where $y_{it}$ is the outcome for individual $i$ at time $t$, $\alpha_i$ and
$\lambda_t$ are individual and year-month fixed effects, respectively, and
$X_{it}$ is a vector of individual and time varying controls. $D_{its}$ equals
1 if, in period $t$, individual $i$ is $s$ months away from signing up to the
app. The set of $\beta_s$ coefficients measure the effect of treatment $s$
periods away from treatment, which is what we are interested in.

We omit the relative period indicator for period $s = -1$ because we need to
omit one relative period indicator to avoid perfect collinearity among the
period indicators, and we choose the last pre-treatment period because it
serves as a natural benchmark against which to compare the outcomes in other
periods.\footnote{As \citet{sun2021estimating} point out, there are two sources
of perfect multicollinearity when estimating a fully dynamic model (i.e. one
including all possible lags). The first results from all relative period
indicators summing to 1 in each period, so that the entire set of relative period
dummies across all time periods is perfectly multicollinear. We deal with this
by excluding the indicator for $s = -1$. The second issue arises from the fact
that for initial treatment period $E_i$, $t = s + E_i$. We deal with this issue
by "trimming" our sample to be balanced in relative periods by only using
data from relative periodl $\{-6, 5\}$. Both of these approaches are standard
in the empirical literature.}


\citet{sun2021estimating} define an event study design as a staggered adoption
design where units are treated at different times and where there may or may
not be never treated units. In our case, we have no never treated units, and
treatment is absorbing in that once a unit is treated they will also be treated
in all subsequent units.\footnote{We cannot rule out that some users who
stopped using the app and closed their account rejoined later on, in which case
they would appear in our dataset as a new user. However, we can plausibly
assume that such cases are rare.} 

Setup:
\begin{itemize}
    \item We observe $N+1$ unites for $T+1$ periods and, for each
        $i\in\{0,\ldots, N\}$ and $t\in\{0,\ldots,T\}$ observe outcome $y_{it}$
        and treatment status $D_{it}\in\{0, 1\}$, where $D_{it}$ equals 1 if unit
        $i$ is treated in period $t$ and 0 otherwise.

    \item We can uniquely characterise treatment paths by the time period of
        initial treatment, denoted as $E_i = min\{t: D_{it} = 1\}$.

    \item We can group units into cohorts $e \in \{0,\ldots, T\}$, where units
        in cohort $e$ were all first treated at time $e$, so that $\{i: E_i =
        e\}$.

    \item We define $y^e_{it}$ as the potential outcome in period $t$ if unit
        $i$ was first treated in period $e$.

\end{itemize}


Assumptions:
\begin{itemize}
    \item Observations $\{y_{it}, D_{it}\}_{t=0}^T$ are independent.

    \item A1: parallel trends: difference in baseline outcomes over time do
        not differ between treatment cohorts. Not obviously violated in our
        context. Early adopters might differ from late adopters, but difference
        might plausibly be constant over time. - We don't have never treated
        units, so Ashenfelter dip scenario is not a problem, even though we
        seem to observe something like this in discretionary spending graph
        (increase in disc spend before signup)

    \item A2: no anticipatory behaviour. Plausibly violated if people are
        motivated to save more and start doing so even before app use. Can test
        for whether there is a peak prior to signup. Because our units have
        private knowledge about future of treatment path (their intention to
        reduce spending and save more and sign up to an app), this might be
        violated. The trajectories of discret spend and net inflows are
        conflicting on this, though, suggesting that they increases discret
        spend in runup to app use (which might provide motivation to eventually
        sign up) but also might have increased net savings slightly.

    \item A3: treatment effect homogeneity across all cohorts and all relative
        periods. (Note: treatment effects can be dynamic, but need to be the
        same across cohorts). We could test for this.

\end{itemize}


Notes:
\begin{itemize}

    \item Comparison: pre vs post signup within each individual.

    \item Assumption: there are no time-varying unobserved effects that affect
        both y and D (formally: $E[Du] = 0$, since u is by definition
        correlated with y.

    \item Discussion: there is something that made the individual sign up in
        the first place, and it might well be an individual level shock that we
        don't observe (unexpected large expense, loss of job, exposure to
        something that motivates saving or change in financial behaviour).

    \item See \citet{imai2021use} for problems with twfe

\end{itemize}


\begin{table}[htbp]
   \centering
   \tiny
   \begin{threeparttable}[b]
      \caption{\label{tab:reg_dynamic} Dynamic results}
      \begin{tabular}{lcccccccc}
         \tabularnewline \midrule \midrule
         Dependent Variables: & \multicolumn{4}{c}{Net-inflows} & \multicolumn{4}{c}{Discretionary spend}\\
         Model:                          & (1)               & (2)               & (3)               & (4)               & (5)                & (6)              & (7)                & (8)\\  
         \midrule
         \emph{Variables}\\
         Months to/since app use $=$ <-6 & -110.84$^{***}$   & -103.66$^{*}$     & -98.69$^{**}$     & -89.62$^{*}$      & -125.40$^{***}$    & -78.16$^{***}$   & -118.87$^{***}$    & -59.75$^{***}$\\   
                                         & [-194.27; -27.42] & [-208.63; 1.30]   & [-196.09; -1.30]  & [-195.15; 15.91]  & [-141.55; -109.25] & [-90.83; -65.49] & [-133.11; -104.63] & [-75.32; -44.18]\\   
         Months to/since app use $=$ -5  & -35.38            & -33.48            & -32.47            & -29.27            & -75.75$^{***}$     & -54.58$^{***}$   & -70.92$^{***}$     & -45.70$^{***}$\\   
                                         & [-147.30; 76.55]  & [-104.30; 37.33]  & [-157.19; 92.26]  & [-101.94; 43.41]  & [-97.42; -54.08]   & [-75.39; -33.77] & [-86.97; -54.88]   & [-60.92; -30.48]\\   
         Months to/since app use $=$ -4  & -84.56            & -72.77            & -83.59            & -70.73            & -57.64$^{***}$     & -35.26$^{***}$   & -55.23$^{***}$     & -29.49$^{***}$\\   
                                         & [-196.42; 27.31]  & [-210.15; 64.60]  & [-209.56; 42.39]  & [-199.69; 58.23]  & [-79.30; -35.98]   & [-51.02; -19.50] & [-70.63; -39.82]   & [-42.97; -16.00]\\   
         Months to/since app use $=$ -3  & -71.07            & -62.18            & -70.38            & -60.24            & -30.29$^{***}$     & -14.47           & -28.66$^{***}$     & -9.71\\   
                                         & [-182.90; 40.75]  & [-170.79; 46.43]  & [-199.96; 59.20]  & [-173.35; 52.87]  & [-51.94; -8.63]    & [-32.82; 3.87]   & [-43.89; -13.44]   & [-28.77; 9.36]\\   
         Months to/since app use $=$ -2  & -122.75$^{**}$    & -119.55$^{**}$    & -122.03$^{**}$    & -118.23$^{**}$    & -4.43              & -1.01            & -2.94              & 1.77\\   
                                         & [-234.56; -10.93] & [-225.23; -13.87] & [-243.52; -0.54]  & [-216.49; -19.96] & [-26.08; 17.22]    & [-18.48; 16.47]  & [-17.65; 11.78]    & [-13.53; 17.06]\\   
         Months to/since app use $=$ 0   & -21.42            & -11.68            & -26.70            & -15.70            & -45.13$^{***}$     & -31.21$^{***}$   & -51.60$^{***}$     & -36.62$^{***}$\\   
                                         & [-133.23; 90.40]  & [-111.40; 88.05]  & [-154.64; 101.24] & [-113.59; 82.19]  & [-66.78; -23.48]   & [-47.29; -15.14] & [-66.74; -36.47]   & [-51.09; -22.15]\\   
         Months to/since app use $=$ 1   & -58.47            & -62.50            & -62.12            & -63.71            & -40.35$^{***}$     & -34.83$^{***}$   & -44.46$^{***}$     & -38.80$^{***}$\\   
                                         & [-170.30; 53.36]  & [-182.33; 57.34]  & [-189.27; 65.03]  & [-176.44; 49.01]  & [-62.01; -18.70]   & [-53.64; -16.02] & [-60.03; -28.88]   & [-57.74; -19.87]\\   
         Months to/since app use $=$ 2   & -122.16$^{**}$    & -128.48$^{**}$    & -124.56$^{**}$    & -128.78$^{**}$    & -27.21$^{**}$      & -34.63$^{**}$    & -30.15$^{***}$     & -39.25$^{***}$\\   
                                         & [-234.00; -10.31] & [-241.72; -15.24] & [-245.46; -3.66]  & [-240.72; -16.84] & [-48.87; -5.56]    & [-61.44; -7.83]  & [-46.11; -14.18]   & [-62.98; -15.53]\\   
         Months to/since app use $=$ 3   & -55.31            & -66.69            & -57.53            & -66.80            & -27.30$^{**}$      & -43.86$^{***}$   & -30.16$^{***}$     & -49.35$^{***}$\\   
                                         & [-167.18; 56.55]  & [-184.08; 50.71]  & [-176.91; 61.85]  & [-174.31; 40.72]  & [-48.96; -5.64]    & [-64.72; -22.99] & [-46.04; -14.28]   & [-66.78; -31.91]\\   
         Months to/since app use $=$ 4   & -85.94            & -97.30            & -87.90            & -96.92            & -23.89$^{**}$      & -37.71$^{***}$   & -26.92$^{***}$     & -43.08$^{***}$\\   
                                         & [-197.82; 25.94]  & [-219.28; 24.68]  & [-207.50; 31.69]  & [-219.67; 25.83]  & [-45.55; -2.22]    & [-60.17; -15.24] & [-42.85; -10.00]   & [-58.87; -27.29]\\   
         Months to/since app use $=$ 5   & 46.03             & 33.02             & 44.40             & 34.50             & -25.47$^{**}$      & -36.66$^{***}$   & -28.84$^{***}$     & -41.60$^{***}$\\   
                                         & [-65.87; 157.93]  & [-100.42; 166.46] & [-76.21; 165.01]  & [-95.93; 164.93]  & [-47.13; -3.80]    & [-56.31; -17.02] & [-45.14; -12.53]   & [-59.75; -23.46]\\   
         Months to/since app use $=$ >5  & -39.39            & -63.30            & -40.34            & -62.71            & -13.55             & -41.94$^{***}$   & -22.44$^{***}$     & -61.44$^{***}$\\   
                                         & [-125.16; 46.38]  & [-171.70; 45.11]  & [-138.79; 58.11]  & [-176.22; 50.79]  & [-30.16; 3.05]     & [-59.32; -24.57] & [-37.23; -7.65]    & [-80.30; -42.58]\\   
         Month income                    & 0.07$^{***}$      & 0.07$^{***}$      & 0.08$^{***}$      & 0.08$^{***}$      & 0.02$^{***}$       & 0.02$^{***}$     & 0.01$^{***}$       & 0.01$^{***}$\\   
                                         & [0.06; 0.08]      & [0.05; 0.09]      & [0.05; 0.10]      & [0.05; 0.11]      & [0.02; 0.02]       & [0.02; 0.03]     & [0.01; 0.02]       & [0.00; 0.01]\\   
         Month spend                     & -0.12$^{***}$     & -0.12$^{***}$     & -0.16$^{***}$     & -0.16$^{***}$     & 0.16$^{***}$       & 0.16$^{***}$     & 0.12$^{***}$       & 0.12$^{***}$\\   
                                         & [-0.13; -0.11]    & [-0.14; -0.10]    & [-0.18; -0.14]    & [-0.19; -0.13]    & [0.16; 0.16]       & [0.15; 0.17]     & [0.11; 0.12]       & [0.11; 0.12]\\   
         Active accounts                 & 36.80$^{***}$     & 36.50$^{***}$     & 67.24$^{***}$     & 67.15$^{***}$     & 42.73$^{***}$      & 41.25$^{***}$    & 78.73$^{***}$      & 72.40$^{***}$\\   
                                         & [27.80; 45.80]    & [22.39; 50.60]    & [49.37; 85.11]    & [45.29; 89.01]    & [40.98; 44.47]     & [38.24; 44.26]   & [73.58; 83.88]     & [67.44; 77.35]\\   
         Intercept                       & 93.70$^{**}$      &                   &                   &                   & 276.72$^{***}$     &                  &                    &   \\   
                                         & [6.98; 180.42]    &                   &                   &                   & [259.93; 293.52]   &                  &                    &   \\   
         \midrule
         \emph{Fixed-effects}\\
         Year-month                      &                   & Yes               &                   & Yes               &                    & Yes              &                    & Yes\\  
         User ID                         &                   &                   & Yes               & Yes               &                    &                  & Yes                & Yes\\  
         \midrule
         \emph{Fit statistics}\\
         Observations                    & 148,932           & 148,932           & 148,932           & 148,932           & 148,932            & 148,932          & 148,932            & 148,932\\  
         R$^2$                           & 0.00892           & 0.00944           & 0.03993           & 0.04044           & 0.41920            & 0.42596          & 0.66235            & 0.67014\\  
         Within R$^2$                    &                   & 0.00881           & 0.00945           & 0.00933           &                    & 0.41085          & 0.28520            & 0.23755\\  
         \midrule \midrule
         \multicolumn{9}{l}{\emph{Signif. Codes: ***: 0.01, **: 0.05, *: 0.1}}\\
      \end{tabular}
   \end{threeparttable}
\end{table}




\begin{figure}[H]
    \centering
    \includegraphics[width=\linewidth]{\figdir/reg_dynamic.png}
    \caption{Dynamic results}%
    \label{fig:reg_dynamic}
\end{figure}


\subsection{Decomposing inflows and outflows}%
\label{sub:decomposing_inflows_and_outflows}

\begin{figure}[H]
    \centering
    \includegraphics[width=.7\textwidth]{\figdir/reg_decomp_inout}
    \caption{Decomposing inflows and outflows}%
    \label{fig:reg_decomp_inout}
\end{figure}


\subsection{Decomposing intensive and extensive margins}%
\label{sub:decomposing_intensive_and_extensive_margins}

\begin{figure}[H]
    \centering
    \includegraphics[width=.7\textwidth]{\figdir/reg_decomp_intext}
    \caption{Decomposing intensive and extensive margin}%
    \label{fig:reg_decomp_intext}
\end{figure}


\subsection{Matching}%
\label{sub:alternative_matching_method}

Control group design:
\begin{itemize}

    \item We only have data for a self-selected group of people who choose to
        use the app. This has a couple consequences:

    \item By virtue of signing up to an app that helps them manage
        their money, these users are different from those who don't
        sign up. As a result, we are unable to answer the question of
        whether app use helps the average person in the
        population as a whole save more.\footnote{One way to get closer
            to that answer is to re-weight our sample on observable
            demographic variables so as to match the UK population as a
            whole. But our sample differs from the population as a
            whole both is ways that are observable (demographic
            variables) and unobservable (self-awareness that they need
            help managing their money, cognitive resources to engage
            with the app, motivation to do so). Re-weighting would only
            help us deal with the first of these.}

    \item Even among people who do eventually sign up to the app,
        hte decision when to do so is unlikely to be random --
        \textit{something} makes them sign up at the particular
        point in time they do and not before or after. If we think
        of this factor as ``motivation to save more'', then said
        motivation is inextricably linked with the decision to sign
        up so that we cannot differentiate between the two.

    \item Hence: due to the first point above, we cannot estimate
        an ATE (effect of app use on the average citizen), and due
        to the second point we also cannot estimate a pure ATT
        (effect of app use on users). Instead, our estimated effect
        of app use captures the effect of being motivated to save
        more and using the app to do so.\footnote{One way to get a
            step closer to ATT would be to find a variable that
        correlates with ``motivation to save'' and control for it /
    match on it.}

    \item This is true for both our matched DiD and our TWFE
        design. While these two approaches use a different
        counterfactual to estimate the effect of app use
        (behaviour of a matched control in the case of DiD and
        extrapolating within-user pre-signup behaviour in the case
        of TWFE), neither can help us with the fact that the
        decision to sign up is likely correlated with the
        time-varying unobserved effect ``motivation to save more''.

\end{itemize}

DiD:
\begin{itemize}

    \item We use a difference-in-differences design to estimate the effect of
        app use. Because we do not have a separate control group, we use the
        per-signup data of Money Dashboard users as control periods and use
        matching to find comparable control user for each tretment user.

    \item To do this, we use the matching estimator for panel data proposed by
        \citet{imai2021matching}. Following paper, we conduct the following
        steps:

    \item For each treated observation, we find a set of control observations
        with that share the same treatment history for a period of $L$ periods
        before the treatment and $F$ periods after the treatment. In our
        baseline specification, we rely on a year's worth of data around the
        treatment period and set $L=6$ and $F = 0, 1, 2, 3, 4, 5$.

    \item Identification assumption is that potential outcomes only depend on
        treatment status of the past L periods. In general, this means that if
        treatment has a cumulative effect over time, the full effect is reached
        after L periods. In our context, this means that any effect on savings
        behaviour from usign the app is fully realised after L periods. (I
        think this means that if we look at the treatment effect for F periods
        forward, the effect should not become stronger after F = L).

\end{itemize}

DiD identification assumptions:
\begin{itemize}
    \item No spillover effects: the potential outcome of unit $i$ at time $t$
        is independent of the treatment status of other units. This is violated
        if a user's partner or friends also use the app and, through sharing
        their experiences or motivations, influence the user's savings
        behaviour.

    \item Carryover effects no longer than $L$ periods: a user's potential
        outcome in period $t$ is independent of treatment status in periods
        more than $L$ periods ago. Given that we are dealing with an absorbing
        treatment, this is not a very strong assumption in our context, and we
        choose $L$ based on what we think is an informative number of periods
        to observe pre-app use behaviour.\footnote{An absorbing treatment is
            one that cannot be reversed, and hence we only change from
            untreated to treated once.}.

    \item Parallel trends: the spending trajectory between treated and control
        units would have continued to be parallel if the treated unit hadn't
        started using the app. This is violated whenever an intended change in
        savings behaviour also provided the impetus for the user to start using
        the app, which is likely to occurr frequently. To the extent this is
        the case, we have an ommitted variable ``motivation to save more'',
        which both changes the user's savings behaviour and their treatment
        status. Because of this, what we are measuring is not a pure ATT of app
        use -- the effect of app use on savings over and above the change
        precipitated by a change motivation -- but the effect of app use for
        users motivated to save more.

\end{itemize}


\begin{figure}[H]
    \centering
    \caption{Match set examples}%
    \includegraphics[width=\linewidth]{\figdir/matchset_examples.png}
    \includegraphics[width=\linewidth]{\figdir/matchset_examples_new.png}
    \label{fig:matchset_examples}

    \fignote{\textwidth}{}

\end{figure}


\begin{figure}[H]
    \centering
    \caption{Distribution of size of matched control units}%
    \includegraphics[width=0.8\linewidth]{\figdir/hist_matchset_size.png}
    \label{fig:hist_matchset_size}

    \fignote{\textwidth}{In the first step of the matching proceedure, each
        user gets assigned a set of potential control users that share the same
        treatment history for a specified number of periods before the user
        signs up to teh app (6 months, in our baseline specification), but that
        do not sign up themselves for a specified number of periods after the
        treatment user has signed up (another 6 months, in our baseline
        specification). The figure shows the distribution of the sizes of these
        sets of potential control users. The pink vertical bar on the left
    shows to count of users for whom no control users cound be found.}

\end{figure}

\begin{figure}[H]
    \centering
    \caption{Covariance balance}%
    \includegraphics[width=0.8\linewidth]{\figdir/covar_balance.png}
    \label{fig:covar_balance}

    \fignote{\textwidth}{Average covariate standard deviation between treatment and
        control units for each pre-treatment period using the entire set of
        potential controls on the left and, on the right, the refined set of
        controls, which, in our baseline specification, consists of the nearest
        neighbour match based on the propensity score.}

\end{figure}

\begin{figure}[H]
    \centering
    \caption{Matching estimates}%
    \includegraphics[width=0.8\linewidth]{\figdir/match_estimates.png}
    \label{fig:match_estimates}

    \fignote{\textwidth}{}

\end{figure}

Is estimate causal?
\begin{itemize}
    \item \citet{king2006dangers} show that there are four sources of bias
        (ommitted variable, posttreatment, interpolation, extrapolation).
    
    \item Discuss each in turn to argue that effect is causal (for our population
        of interest, which are people signing up to MDB). 
\end{itemize}


\subsection{Alternative window lengths}%
\label{sub:alternative_window_lengths}

\subsection{Subgroups}%
\label{sub:subgroups}

To analyse which groups benefit most from adopting Money Dashboard, we split
our sample by gender, generation, income quartiles, and pre-adoption savings
behaviour.

We define generations as follows: boomers were born between 1946 and 1964, Gen
X between 1965 and 1980, Millennials between 1981 and 1996, and Gen Z after
1997.\footnote{Based on age ranges provides by
    \href{https://www.beresfordresearch.com/age-range-by-generation/}{Beresford
Research}.}

Subgroup analysis: same Fig an Tab as in main analysis, but with line for each
subgroup. One figure for each of: gender, generations, income terciles,
per-adoption average savings tercile (inspired by \citet{carlin2017fintech},
see Fig 5 and Table 4).

See also section 6 in \citet{gargano2021goal}


\subsection{Alternative outcome variables}%
\label{sub:alternative_outcome_variables}

Look at netflows scaled by income



\section{Results}%
\label{sec:results}

Main results: Figure and associated table akin to sakaguchi2022default Figure 2
and Table 6.

\subsection{Descriptive analysis}%
\label{sub:descriptive_analysis}

\subsection{Difference-in-difference analysis}%
\label{sub:difference_in_difference_analysis}

\subsection{Subgroups}%
\label{sub:subgroups}

To analyse which groups benefit most from adopting Money Dashboard, we split
our sample by gender, generation, income quartiles, and pre-adoption savings
behaviour.

We define generations as follows: boomers were born between 1946 and 1964, Gen
X between 1965 and 1980, Millennials between 1981 and 1996, and Gen Z after
1997.\footnote{Based on age ranges provides by
    \href{https://www.beresfordresearch.com/age-range-by-generation/}{Beresford
Research}.}

Subgroup analysis: same Fig an Tab as in main analysis, but with line for each
subgroup. One figure for each of: gender, generations, income terciles,
per-adoption average savings tercile (inspired by \citet{carlin2017fintech},
see Fig 5 and Table 4).

% % !TEX root = ../eval.tex

\section{Discussion}%
\label{sec:discussion}

Limitations:
\begin{itemize}
    \item Can't say whether increase in savings was achieved by going into
        debt elsewhere
\end{itemize}

\section{Discussion}%
\label{sec:discussion}


\newpage
\printbibliography
\newpage

% \input{sections/appendix.tex}
\appendix

\section{Money Dashboard application}%
\label{sec:money_dashboard_application}

\begin{figure}[htpb]
    \centering
    \caption{Money Dashboard website screenshot}%
    \includegraphics[width=0.8\linewidth]{\figdir/mdb_website}
    \label{fig:mdb_website}
\end{figure}
\fignote{\textwidth}{Screenshot from the top of the Money Dashboard website, at
\href{https://www.moneydashboard.com}{moneydashboard.com}, accessed on 29 April
2022.}

\section{Variable definitions}%
\label{sec:variable_definitions}

Complete steps from raw data to variables used in analysis, with links to code
on Github.


\section{Alternative control group designs}%
\label{sec:alternative_control_group_designs}

\subsection{Alternative window lengths}%
\label{sub:alternative_window_lengths}

Show results for 12 months on either end.

\subsection{Alternative matching method}%
\label{sub:alternative_matching_method}

\subsection{Post treatment periods as control}%
\label{sub:post_treatment_periods_as_control}

\subsection{Synthetic controls}%
\label{sub:synthetic_controls}

\subsection{Classic two-way fixed-effects model}%
\label{sub:classic_two_way_fixed_effects_model}

As discussed in \citet{imai2021matching}.

\subsection{Alternative two-way fixed-effects model}%
\label{sub:alternative_two_way_fixed_effects_model}

Use fixest implementation of \citet{sun2021estimating}.





See \citet{abadie2021penalized} for how to use synthetic controls for
disaggregated data.




\end{document}
