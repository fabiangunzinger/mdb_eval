% !TEX root = ../eval.tex

\section{Variable construction}%
\label{sec:variable_construction}

The table below describes the construction and rationale for including of all
variables used. The code used to construct the variables is available on
\href{https://github.com/fabiangunzinger/mdb_eval/blob/d094f8cd364f64bbe3d4e644abbff726af86de2f/src/data/aggregators.py}{GitHub}.

\begin{table}[htpb]
    \centering\scriptsize
    \caption{Variable construction}
    \label{tab:vars}
    \begin{tabularx}{\textwidth}{>{\raggedright\arraybackslash}X
        >{\raggedright\arraybackslash}X>{\raggedright\arraybackslash}X}
    \hline\hline
    Variable (name in dataset) & Definition  & Rationale \\
    \hline\\
    \multicolumn{3}{c}{\textbf{Primary outcome}}\\\\

    \multicolumn{3}{c}{\textbf{Covariates}}\\\\

    New loan dummy (\textit{new\_loan})&
    Dummy variable equal to 1 if user takes out a new loan. Calculated positive inflows
    of funds tagged as ``loan''.&
    Might increase (additional funds) or decrease (need to repay) propensity to
    save in month of takeout and lower propensity to save in the future due to
    need to repay.\\

    Unemployment benefits dummy (\textit{unemp\_benefits})&
    Dummy variable equal to 1 if user has inflow of funds tagged as ``job
    seeker benefits''.&
    Might lower a user's ability to save but increase their need for a money
    management app.\\

    Monthly income (\textit{month\_income})&
    Average monthly income in a calendar year, calculated as the sum of all
    credits tagged income payments in said year divided by 12.&
    Income may alter the need and ability to save and correlate with cognitive
    characteristics that alter a person's propensity to use a money management
    app.\\

    \hline\hline
    \end{tabularx}
\end{table}

