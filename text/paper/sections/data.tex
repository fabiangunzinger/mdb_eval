% !TEX root = ../eval.tex

\section{Data}%
\label{sec:data}

\subsection{Dataset description}%
\label{sub:dataset_description}

I use data from Money Dashboard (MDB), a UK-based financial aggregation app
that allows users to link accounts from different banks to obtain an integrated
view of their finances in order to help them to reduce their spending and
increase their savings. Figure~\ref{fig:mdb} shows a screenshot from MDB's
website (left) and an example of its user interface (right). The complete
dataset contains more than 500 million transactions made between 2012 and June
2020 by about 270,000 users, and provides information such as date, amount, and
description of the transaction as well as account and user-level information.

\begin{figure}[h]
    \centering
    \caption{Money Dashboard website and user interface}%
    \includegraphics[width=0.49\linewidth]{\figdir/mdb_website}
    \includegraphics[width=0.49\linewidth]{\figdir/mdb_ui}
    \label{fig:mdb_website}
    \fignote{\textwidth}{Screenshot from the top of the Money Dashboard website
    (left) and example of its user interface (right), illustrating some of the
transaction grouping and budgeting functions.} 
\end{figure}

Crucially, for this paper, MDB can access up to three years of historic data
for each linked account. Figure~\ref{fig:treatplot} shows how we can leverage
pre-signup data provided in the MDB data to construct valid control groups.
Each row of cells shows data for one of 200 randomly selected users, with red
cells indicating data from pre-signup months and blue cells indicating data
from post-signup months. Signup thus occured sometime during the first blue
coloured month, and we can use users who have not yet signed up (whose cells
are still red) as a comparison group. In my analysis below, I will use this
data structure to compare the outcomes of users who start using the Money
Dashboard app at a particular point in time with users who have not yet started
using the app at that time.

\begin{figure}[h]
\centering
\caption{Pre and post-signup data availability}%
\includegraphics[width=0.8\linewidth]{\figdir/treatplot_sample_raw.png}
\label{fig:treatplot}
\fignote{\textwidth}{Each horizontal line shows the observed pre and post
    signup periods in blue and red, respectively, for one of 200 randomly
    selected users. The faint vertical white lines indicate month borders,
    whitespace indicates periods in which we do not observe the user. To
the left of the observed period, this is because the app cannot access data
before that point when the user signs up; to the right, because they have
stopped using the app.}
\end{figure}

The data's main advantage for the study of consumer financial behaviour is that
it allows us to observe all savings and spending transactions for users who
linked all their financial accounts. This is important since it means that we
can be sure that any changes we observe in some of a user's account are not
offset by changes in unliked accounts.\footnote{For an example where debt-reduction in
one set of accounts is offset by an increase in debt by another set of account,
see \citet{medina2021side}.} Furthermore, data is collected automatically and
in real-time rather than through surveys that collect data with a time-lag and
often rely on consumers's ability and willingness to provide accurate
information.

The data's main limitation is that because users's self select into using MDB,
the sample is not representative of the wider UK population: it is well
documented that FinTech app users are more likely to be male, younder, and
higher-income earners than the average person \citet{carlin2019generational}.
Also, as pointed out in \citet{gelman2014harnessing}, a willingness to share
financial information with a third party might not only select on demographic
characteristics, but also for an increased need for financial management or a
higher degree of financial sophistication. However, because, as discussed in
the introduction, the aim of this paper is to assess the effect of MDB on
people who choose to use it, the lack of representativeness is not an
issue.\footnote{For an example of how re-weighing can be used to mitigate the
non-representative issue, see \citet{bourquin2020effects}.} A second limitation
is that while we can observe user's complete financial behaviour if they add
all their financial accounts to the app, it is not trivial to distinguish
between users who do and do not do that. I address this challenge in the sample
selection process documented below. A third issue is that while the app is able
to classify many transactions into types, it misclassifies some transactions
and cannot classify others altogether. I address this as part of the cleaning
process documented below.

\subsection{Preprocessing}%
\label{sub:preprocessing}

\paragraph{Data cleaning:}%
\label{par:data_cleaning}

I use the dataset described above for a number of projects, and perform a
number of steps to create a minimally cleaned version of the dataset that is
the basis for all such projects. These steps are performed in a dedicated data
repository and not run as part of this project.\footnote{The module with all
cleaning functions is available on
\href{https:/egithub.com/fabiangunzinger/mdb_eval/blob/f51e49c95c5884d2dc417be23921a8acd85aec9d/src/data/clean.py}{Github}.}
Here, I briefly describe the main cleaning steps and their rationale. I drop
all transactions with a missing description string because these cannot be
categoriesed, and all transactions that are not automatically categoriesed by
the app. Dropping these transactions makes it likely that I will underestimate
amounts spent and saved, but minimises the risk of incorrectly classified
transactions. I also group transactions into transfer, spend, and income
subgroups, following \citet{muggleton2020evidence} to define spend subgroups
and \citet{hacioglu2021distributional} to define income subgroups.\footnote{
The precise list used to classify transactions is available on
\href{https://github.com/fabiangunzinger/mdb_eval/blob/92af366d4c4052cc7a7f78a6178086de8ecdfb75/src/data/txn_classifications.py}{Github}.}
Finally, I classify as duplicates and drop transactions with identical user ID,
account ID, date, amount, and transaction description. This will drop some
genuine transactions, such as when a user buys two identical cups of coffees at
the same coffee shop on the same day. However, data inspection suggests that in
most cases, we remove genuine duplicates.

To minimise the influence of outliers, I winsorise all variables at the 1
percent level or -- if we winsorise on both ends of the distribution -- at the
0.5 percent level.\footnote{The code that performs the winsorisation are
    available on
\href{https://github.com/fabiangunzinger/mdb_eval/blob/d04fe186bb5cca884af2b7c1c7ad429674ef701d/src/data/transformers.py}{Github}.}
I rely on winsorisation (replacing top values with percentile values) instead
of trimming (replacing top values with missing values) because data inspection
suggests that in most cases, very large (absolute) values are not the result of
data errors, which would call for trimming, but reflect genuine outcomes, which
makes winsorising appropriate because it leaves these observations in the data
while lowering their leverage to influence results.


\paragraph{Sample selection:}%
\label{par:sample_selection_}

The three main goals of sample selection are to select a sample of users for
whom I can be reasonably certain to observe all relevant financial
accounts,\footnote{Relevant accounts include all current, savings, and
credit-card accounts, but exclude long-term savings and investment accounts.}
account histories of at least 12 months, and who are not using MDB for business
purposes. Table~\ref{tab:selection} lists the precise conditions I apply to
implement these criteria and their effect on sample size.\footnote{The code
that implements the selection criteria is available on
\href{https://github.com/fabiangunzinger/mdb_eval/blob/main/src/data/selectors.py}{Github}.}

In my main analysis, I show effects of app use for the 12-months period from 6
months before and 5 months after MDB signup, treating the month of signup as
period 0. To ensure that results are not affected by the number of accounts we
observe for an individual, it is thus critical that we can observe at least 12
months of history for all accounts a user adds to the platform. This ensures
that in the extreme case where a user adds an account they had used for some
time in the fifth month after they signed up, I observe all transactions on
that account throughout the 12-month period of interest.\footnote{For example:
    if a user has made monthly payments of \pounds100 into a savings account
    for two years by the time they sign up to MDB but links that account five
    months after joining, we can only observe the historical payments if MDB
    can access at least 12 months of history. If this is not the case, and MDB
    can only, say, access 6 months of history, we would erroneously conclude
that the user started saving \pounds100 more starting in the month they signed
up.} Data exploration suggests that all major banks start providing 12 months
of historical data from April 2017 onwards, which is why include only users who
sign up after that data.\footnote{The Jupyter Notebook containing the data
exporation is available on
\href{https://github.com/fabiangunzinger/mdb_eval/blob/371493bd78870cab7a303edb70687287e5bca4a9/notebooks/available_account_history.ipynb}{Github}.}

To ensure that I can be reasonably certain to observe users have added all
their financial accounts to the app, I restrict our sample to users with at
least one savings and current account, with an annual income of at least
\pounds5,000, and a minimum of 10 transactions and a spend of \pounds200 every
month. To remove users who might use the app for business purposes, I drop
users with more than 10 active accounts in any given month.

Finally, I drop test users and users that are not of working age (younger than
18 or older than 65) because these groups might have objectives other than to
reduce spending and increase savings.\footnote{We cannot identify test users
    precisely, but drop users who signed up prior to or during 2011, the first
year the app was in operation.} And I retain only users for whom we can observe
the complete set of demographic covariates (gender, age, region) to ensure all
users can be used throughout the analysis. These steps do impact the sample
size significantly.

\begin{table}
\centering\footnotesize
\caption{Sample selection}\label{tab:selection}
\begin{tabular}{lrrrr}
\toprule
                                       &   Users & User-months &        Txns & Txns (m\pounds) \\
\midrule
                            Raw sample & 271,856 &   7,948,520 & 662,112,975 &         124,573 \\
Annual income of at least \pounds5,000 &  92,913 &   2,546,058 & 243,279,442 &          44,616 \\
          At least one savings account &  54,885 &   1,645,698 & 166,547,361 &          32,920 \\
            At least 12 months of data &  47,888 &   1,604,982 & 163,679,001 &          32,436 \\
  At least 6 months of pre-signup data &  30,597 &     848,878 &  88,938,732 &          18,419 \\
  Monthly spend of at least \pounds200 &  16,421 &     451,959 &  52,289,786 &          11,351 \\
           At least 10 txns each month &  15,405 &     420,857 &  49,333,260 &          10,521 \\
      Complete demographic information &  12,318 &     347,132 &  40,993,232 &           8,541 \\
                           Working age &  12,085 &     339,487 &  40,260,306 &           8,232 \\
                          Final sample &  12,085 &     339,487 &  40,260,306 &           8,232 \\
\bottomrule
\end{tabular}

\tabnote{.95\textwidth}{Number of users, user-months, transactions, and
transaction volume in millions of British Pounds left in our sample after each
sample selection step.}
\end{table}


\subsection{Summary statistics}%
\label{sub:summary_statistics}

Figure~\ref{fig:sample_description} provides a view of some salient sample
characteristics. Panels A and B show that while distributions of disposable
income and total spend in 2019 broadly mirror that of the ONS Living Cost and
Food Survey (LCFS) data, the MDB data tends to slightly underestimate incomes
and overestimate spending.\footnote{I accessed the LCFS data via the UK Data
Service at the following url:
\url{https://beta.ukdataservice.ac.uk/datacatalogue/studies/study?id=8686}.}
Given that our sample is likely biased towards high-earners, as discussed
above, we would expect both income and spend to be higher than in the LCFS. Two
likely caveats to the data probably create discrepancies relative to the LCFS
data. First, as discussed in Section~\ref{sub:preprocessing}, I drop
unclassified transactions from the data, which biases both income and spending
downwards (i.e. spending would probably be even higher compared to the LCFS).
Second, it is likely that it is more challenging to automatically classify
income transactions than spend transactions, which might create an additional
downward bias on the income distribution.\footnote{\citet{bourquin2020effects}
    present an alternative algorithm to identify income transactions and find
that this leads to higher estimated incomes. I do not use their algorithm
because I only use disposable income as a covariate to capture relative income
differences between users.} Panels C, D, and E show that, as discussed in
Section~\ref{sub:dataset_description}, the sample is skewed towards users that
are younger, male, and live in the South East. Finally, Panel D shows that most
users use transact with one or two accounts per month.

\begin{figure}[H]
    \centering
    \caption{Sample characteristics}
    \includegraphics[width=\linewidth]{\figdir/sample_description.png}
    \label{fig:sample_description}
    \fignote{\textwidth}{Panels A and B show the distribution of disposable
        income and total spending in 2019, respectively, benchmarked against
        the 2018/19 wave of the ONS Living Cost and Food Survey (LCFS). The
        remaining panels show the data distributions of age, gender, region,
    and the number of active accounts.}%
\end{figure}

Table~\ref{tab:sumstats} provides additional summary statistics. It shows, for
instance, that the average number of monthly transactions is slightly above 100
(with a mean of 112 and a median of 101), that half of all user-month incomes
lie between \pounds1,407 and \pounds3,596, and that inflows into savings
accounts closely mirror outflows. It also suggests that highly discretionary
spend accounts for about 30 percent of total monthly spend

\begin{table}[ht]
\centering\tiny
\caption{Summary statistics}
\label{tab:sumstats}

% Table created by stargazer v.5.2.3 by Marek Hlavac, Social Policy Institute. E-mail: marek.hlavac at gmail.com
% Date and time: Fri, Jul 08, 2022 - 16:36:19
\begin{table}[H] \centering 
  \caption{Summary statistics} 
  \label{tab:sumstats} 
\footnotesize 
\begin{tabular}{@{\extracolsep{5pt}}lccccccc} 
\\[-1.8ex]\hline 
\hline \\[-1.8ex] 
Statistic & \multicolumn{1}{c}{Mean} & \multicolumn{1}{c}{St. Dev.} & \multicolumn{1}{c}{Min} & \multicolumn{1}{c}{Pctl(25)} & \multicolumn{1}{c}{Median} & \multicolumn{1}{c}{Pctl(75)} & \multicolumn{1}{c}{Max} \\ 
\hline \\[-1.8ex] 
user\_id & 503,356.1 & 34,778.6 & 441,309 & 474,770 & 499,754 & 528,755 & 575,009 \\ 
ym & 587.2 & 9.9 & 560 & 580 & 588 & 595 & 606 \\ 
ymn & 201,854.3 & 86.1 & 201,609 & 201,805 & 201,901 & 201,908 & 202,007 \\ 
month & 6.5 & 3.5 & 1 & 3 & 6 & 9 & 12 \\ 
txns\_count & 113.5 & 60.4 & 10 & 71 & 102 & 143 & 335 \\ 
txns\_volume & 18,110.9 & 25,559.3 & 227.1 & 6,036.4 & 10,291.1 & 18,875.1 & 180,795.9 \\ 
month\_income & 2,931.6 & 2,238.4 & 417.3 & 1,510.9 & 2,243.2 & 3,567.6 & 13,016.6 \\ 
inflows & 748.7 & 2,498.0 & 0.0 & 0.0 & 0.0 & 400.0 & 19,353.4 \\ 
outflows & 738.6 & 2,388.1 & 0.0 & 0.0 & 0.0 & 370.0 & 17,933.6 \\ 
netflows & 15.2 & 2,894.2 & $-$19,940.9 & 0.0 & 0.0 & 50.0 & 22,000.0 \\ 
netflows\_norm & $-$0.001 & 1.0 & $-$7.5 & 0.0 & 0.0 & 0.02 & 7.7 \\ 
inflows\_norm & 0.3 & 0.9 & 0.0 & 0.0 & 0.0 & 0.2 & 6.7 \\ 
outflows\_norm & 0.3 & 0.9 & 0.0 & 0.0 & 0.0 & 0.2 & 6.7 \\ 
has\_pos\_netflows & 0.3 & 0.5 & 0 & 0 & 0 & 1 & 1 \\ 
pos\_netflows & 413.9 & 2,078.5 & 0.0 & 0.0 & 0.0 & 50.0 & 22,000.0 \\ 
user\_reg\_ym & 590.8 & 5.4 & 580 & 587 & 590 & 595 & 601 \\ 
t & 0.4 & 0.5 & 0 & 0 & 0 & 1 & 1 \\ 
tt & $-$3.6 & 11.2 & $-$35 & $-$10 & $-$3 & 4 & 26 \\ 
month\_spend & 2,742.6 & 2,613.1 & 200.1 & 1,198.0 & 1,974.8 & 3,288.0 & 16,936.3 \\ 
age & 36.7 & 10.1 & 18 & 29 & 35 & 43 & 65 \\ 
is\_female & 0.4 & 0.5 & 0 & 0 & 0 & 1 & 1 \\ 
is\_urban & 0.8 & 0.4 & 0 & 1 & 1 & 1 & 1 \\ 
region\_code & 4.0 & 3.2 & 0 & 1 & 4 & 6 & 13 \\ 
has\_savings\_account & 1.0 & 0.0 & 1 & 1 & 1 & 1 & 1 \\ 
has\_current\_account & 1.0 & 0.0 & 1 & 1 & 1 & 1 & 1 \\ 
generation\_code & 2.6 & 0.7 & 1 & 2 & 3 & 3 & 4 \\ 
prop\_credit & 0.1 & 0.2 & 0.0 & 0.0 & 0.0 & 0.1 & 1.0 \\ 
dspend & 859.4 & 748.0 & 0.0 & 365.6 & 653.6 & 1,109.4 & 4,236.4 \\ 
dspend\_count & 39.8 & 27.9 & 0 & 21 & 35 & 52 & 465 \\ 
dspend\_mean & 24.7 & 42.1 & 0.2 & 13.8 & 19.5 & 27.6 & 7,508.2 \\ 
accounts\_active & 3.3 & 1.8 & 1 & 2 & 3 & 4 & 10 \\ 
accounts\_total & 5.8 & 2.8 & 2 & 4 & 5 & 8 & 19 \\ 
\hline \\[-1.8ex] 
\end{tabular} 
\end{table} 

\end{table}

