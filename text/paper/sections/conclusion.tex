% !TEX root = ../eval.tex

\section{Conclusion}
\label{sec:conclusion}

In this paper, I test whether using Money Dashboard is associated
with a reduction in discretionary spending and an increase in emergency
savings. I use a new estimator proposed by \citet{callaway2021difference}
that corrects for recently identified problems in two-way fixed effects
estimates.

I find that users reduce their discretionary spend by between \pounds100 and
\pounds150 (11-17\% of average discretionary spend) once they start using the
app and sustain that reduction throughout the six-month post-signup period I
consider. Looking at disaggregated measures of discretionary spend further
shows that the reduction is the result of maintained month-to-month changes in
behaviour rather than one-off cancellations of direct-debit transactions, that
it results from reducing spending on a number of different categories of
purchases rather than a single one, and that it is a result of changes along
the extensive rather than the intensive margin -- users reduce the number of
transactions they make rather than the value of the average transaction.

Interestingly, users do not seem to use these additional funds to build up
emergency savings: net-inflows into savings accounts do not change after
signup. I can also neither find significant increases in flows into investment
and pension accounts or additional savings accounts that are not linked to the
app, not additional loan repayments.

We cannot interpret these results as causal effects because I cannot isolate
the effect of MDB use from possible confounding factors that make users
self-select into using the app in the first place. And we cannot attribute the
effect of MDB use to any specific feature of the app, such as providing easy
access to financial information or goal-setting features.

However, the findings provide novel insight into how people reduce their
spending and are strongly suggestive that using a financial aggregator app can
help people reduce their spending. While -- as my results show -- such a
reduction in spend does not automatically translate into an increase in
emergency savings, they provide an opportunity for such savings. To harness
that opportunity, FinTech apps should experiment with incorporating tools that
make saving easy.

Further research could focus on causally establishing the link between app use
and discretionary spend, as well as testing the effectiveness of various tools
to help users turn those additional funds into savings. Another area that would
be useful to understand is to what extent the effect of app use might depend on
a ``dosage effect'' -- whether the effectiveness of app features is moderated
by the intensity of app use. However, work testing such an effect will have to
isolate dosage of exposure from personal characteristics that make certain
users engage more with the app. Finally, the finding that people may direct
saved funds towards a number of different purposes can be the result of
differences in funds use across people -- different people use funds for
different purposes -- or within people -- people all use the funds for a
variety of different purposes -- or, most plausibly, a combination of the two.
If people spread the additional funds across many different purposes then it
might be the case that they miss out on directing them instead towards the use
with the highest long-term payoff, such as paying downs of credit-card debt or
investing in government-subsidised savings vehicles.\footnote{In the UK, for
  instance, the first £4000 paid into a Lifetime ISA each year are matched by
  the government by a £1000 payment, corresponding to a risk-free interest rate
of 25 percent.} This behaviour would be similar to the finding by
\citet{gathergood2019individual} that -- instead of first paying down debt of
their highest-interest credit card, people tend to allocate repayments across
all their cards in proportion to outstanding debt. Further research
establishing to what extent there is within-person variation of fund use, and
whether such funds are directed towards the most effective uses would provide
further useful insights into savings behaviour, and could serve as the basis
for further development of FinTech apps to help users direct saved funds
towards the most effective users.

