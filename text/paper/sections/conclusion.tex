% !TEX root = ../eval.tex

\section{Conclusion}
\label{sec:conclusion}

In this paper, I test whether using Money Dashboard is associated
with a reduction in discretionary spending and an increase in emergency
savings. I use a new estimator proposed by \citet{callaway2021difference}
that corrects for recently identified problems in two-way fixed effects
estimates.

I find that users reduce their discretionary spend by between \pounds100 and
\pounds150 (11-17\% of average discretionary spend) once they start using the
app and sustain that reduction throughout the six-month post-signup period I
consider. Looking at disaggregated measures of discretionary spend further
shows that the reduction is the result of maintained month-to-month changes in
behaviour rather than one-off cancellations of direct-debit transactions, that
it results from reducing spending on a number of different categories of
purchases rather than a single one, and that it is a result of changes along
the extensive rather than the intensive margin -- users reduce the number of
transactions they make rather than the value of the average transaction.
Interestingly, users do not seem to use these additional funds to build up
emergency savings: net-inflows into savings accounts do not change after
signup. I can also neither find significant increases in flows into investment
and pension accounts or additional savings accounts that are not linked to the
app, not additional loan repayments.

The main limitation of my approach is that I cannot isolate the effect of MDB
use from possible confounding factors that let users to self-select into using
the app in the first place, and that I cannot isolate the effect of individual
components of the app such as information and goal-setting. However, I do find
that app use is associated by a statistically and economically significantly
reducuction of discretionary spend that is sustained for at least six months,
and that seems to be brought about by a reduction in the number of purchases
across a number of spending categories. This provides novel insights into how
people that probably do have some unobserved motivation to reduce their savings
actually achieve this. The additional finding that a reduction in spend is not
accompanied by a commensurate increase in either short-term savings,
investments, or debt-reduction suggests that people use the saved funds for a
variety of different purposes.

The findings also suggest promising directions for future research. First, the
substantial drop in discretionary spend associated with app use does suggest
that MDB and apps like it might have a positive causal impact on financial
outcomes, making it worthwhile to study their effect in more detail in ways
that account for the two limitations mentioned above. Second, the effect of app
use might depend on a ``dosage effect'', so that the behaviour of more active
users is effected more. However, work testing such an effect will have to
isolate dosage of exposure from personal characteristics that make certain
users engage more with the app. Third, the finding that people may direct saved
funds towards a number of different purposes can be the result of differences
in funds use across or within people or a combination of the two. Within people
variation in fund use could suggest the possibility that -- similarly to
allocating credit-card payments towards different cards proportional to
outstanding debt rather than towards highest-interest cards
\citet{gathergood2019individuals} -- people direct additional cash-flows
towards a number of different uses instead of towards those with the highest
payoff, such as paying down credit-card debt or investing in
government-subsidised high-interest investments.\footnote{In the UK, for
    instance, the first \pounds4000 paid into a Lifetime ISA each year are
matched by the government by payment of \pounds1000, corresponding to a
risk-free interest rate of 25 percent.} Further research establishing to what
extent there is within-person variation of fund use, and whether such funds are
directed towards the most effective uses would provide further useful insights
into savings behaviour, and could serve as the basis for further development of
financial aggregator apps to help users direct saved funds towards the most
effective users.


