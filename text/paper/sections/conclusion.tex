% !TEX root = ../eval.tex

\section{Conclusion}
\label{sec:conclusion}

Limitations:
\begin{itemize}
    \item Can't say whether increase in savings was achieved by going into
        debt elsewhere

    \item Limitations: We have more data for users that signed up later. So average user in
        the study is not the average MDB user. If time of signup is mainly
        driven by financial savyness, then study sample is closer to overall
        population than MDB sample (if we rank groups as early joiners > late
        joiners > never joiners in terms of financial sophistication). If,
        however, signup reflects something like openness to newness, then it's
        not necessarily correlated with financial savyness. Either way, we
        might ignore it for now. We could test whether behaviour differs
        between early or late adopters, but that doesn't seem important enough.

    \item We have good reason to believe that the conditional parallel trend
        assumption doesn't hold. For now, I ignore this. But in future work, I
        want to explore this further using approach by
        \citet{rambachan2022more}.

    \item We cannot say whether effect is driven by information or
        goal-setting.

    \item I ignore cash spending and focus on card spending only because cash
        spending cannot be grouped into expense categories and because it
        represents no more than 15 percent of total spend.\footnote{This is an
            upper bound, since cash spending is the sum of all observed ATM
        withdrawals and some of that money will have been used for non-spend
    transactions}.

    \item It's plausible that app has negative effect as it crowds out
use of superior means of budgeting. 


\end{itemize}


Notes:
\begin{itemize}
    \item Maybe not more effective because use of phone makes good decision
        making more difficult due to size of screen and layout and ``distracted
        mindset`` \citet{levi2020mind}.
\end{itemize}
