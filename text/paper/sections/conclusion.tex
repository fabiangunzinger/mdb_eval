% !TEX root = ../eval.tex

\section{Conclusion}
\label{sec:conclusion}

In this paper, I test whether using Money Dashboard is associated
with a reduction in discretionary spending and an increase in ``rainy-day
savings''. I use a new estimator proposed by \citet{callaway2021difference}
that corrects for recently identified problems in two-way fixed effects
estimates.

I find that users reduce their discretionary spend by between \pounds100 and
\pounds150 (11-17\% of average discretionary spend) once they start using the
app and sustain that reduction throughout the six-month post-signup period I
consider. Looking at disaggregated measures of discretionary spend further
shows that the reduction is the result of maintained month-to-month changes in
behaviour rather than one-off cancellations of direct-debit transactions, that
it results from reducing spending on a number of different categories of
purchases rather than a single one, and that it is a result of changes along
the extensive rather than the intensive margin -- users reduce the number of
transactions they make rather than the value of the average transaction.
Interestingly, users do not seem to use these additional funds to build up
``rainy-day savings'': net-inflows into savings accounts do not change after
signup. I can also neither find significant increases in flows into investment
and pension accounts or additional savings accounts that are not linked to the
app, not additional loan repayments.

Limitations:
\begin{itemize}

    \item Can't identify effect of MDB

    \item Can't identify component of MDB (info vs goal-setting)

    \item We have good reason to believe that the conditional parallel trend
        assumption doesn't hold. For now, I ignore this. But in future work, I
        want to explore this further using approach by
        \citet{rambachan2022more}.

    \item Card spend only - no cash.

    \item It's plausible that app has negative effect as it crowds out
use of superior means of budgeting. 


\end{itemize}


Notes:
\begin{itemize}
    \item Maybe not more effective because use of phone makes good decision
        making more difficult due to size of screen and layout and ``distracted
        mindset`` \citet{levi2020mind}.
\end{itemize}
