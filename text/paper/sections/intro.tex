% !TEX root = ../eval.tex

\section{Introduction}%
\label{sec:introduction}

This paper evaluates whether Money Dashboard, a UK-based financial aggregator
app, helps its users reduce their discretionary spend and increase their
``emergency savings'' -- short-term savings intended as a buffer against
unexpected financial shocks.

The question is important because a large number of adults in the UK and the US
do not have enough savings to cover unexpected expenses like car or medical
bills: in the UK, 25 percent of adults would be unable to cover an unexpected
bill of \pounds300 \citep{philipps2021supporting}, while in the US, about 30
percent would be unable to cover a \$400 bill \citep{fed2022economic}. But
while there is a large body of research that studies reasons for why savings
are low, little is known about what could help people save
more.\footnote{Well-documented behavioural biases that help explain undersaving
    are, among others, present bias \citep{laibson1997golden,
    ericson2019intertemporal}, inertia \citep{madrian2001power},
    over-extrapolation \citep{choi2009reinforcement}, and limited self-control
    and willpower \citep{thaler1981economic, benhabib2005modeling,
    fudenberg2006dual, loewenstein2004animal, gul2001temptation}. One danger of
    viewing low savings mainly as a result of behavioural biases is that while
    these biases likely do play some role and designing environments and tools
    to help correct them are thus part of the solution, it is at least
    conceivable that this is an area where the focus on behaviour-level
    solutions distracts from an effort to find more effective society-level
    solutions, a danger inherent in behavioural science research convincingly
    highlighted in \citet{chater2022frame}: if the main problem is that many
people are unable to earn enough to save, then the effectiveness of helping
them manage their low incomes more effectively pales in comparison with efforts
to help them earn more.}

FinTech aggregator apps like Money Dashboard are promising because they provide
easy access to financial information that make it easier to monitor ones
spending and saving, and often also offer tools such as budgeting and the
setting of spending goals. Financial information that
is more easily accessible and is aggregated in ways that help people keep track
of their goals might be beneficial because rational inattention theory predicts
that it makes people more likely to access that information, which, in turn,
might lead to better consumption decisions. Similarly, tools that help with
budgeting and with setting spending goals have the potential to help users make
consumption decisions more in line with their intentions because such tools can
act as commitment devices that -- if users experience disutility from falling
short of their goals -- introduce a cognitive cost to overspending or
undersaving.\footnote{On rational inattention theory, see, for instance,
    \citet{brunnermeier2008wealth, dellavigna2009psychology,
    sims2003implications}. On commitment devices see, among others,
    \citet{thaler1981economic, laibson1997golden, o1999doing} for theoretical
    foundations, and \citet{beshears2016beyond, hsiaw2013goal} for a discussion
of soft commitment devices.}

The nascent literature that studies the effect of FinTech apps on financial
outcomes suggests that these apps can indeed lead to improved financial
outcomes: they have been found to reduce spending by providing users with
information about their spending relative to peers
\citep{dacunto2020crowdsourcing} and offering budgeting options
\citep{lukas2022influence}, to increase savings by offering budgeting options
\citep{gargano2021goal}, and to reduce non-sufficient fund fees by facilitating
access to information \citep{carlin2022mobile}.

In this paper, I specifically test whether using Money Dashboard is associated
with a reduction in discretionary spending and an increase in emergency
savings. I use a new estimator proposed by \citet{callaway2021difference} that
corrects for recently identified problems in two-way fixed effects estimates.

I find that users reduce their discretionary spend by between \pounds100 and
\pounds150 (11-17\% of average discretionary spend) once they start using the
app and sustain that reduction throughout the six-month post-signup period I
consider. Looking at disaggregated measures of discretionary spend further
shows that the reduction is the result of maintained month-to-month changes in
behaviour rather than one-off cancellations of direct-debit transactions, that
it results from reducing spending on a number of different categories of
purchases rather than a single one, and that it is a result of changes along
the extensive rather than the intensive margin -- users reduce the number of
transactions they make rather than the value of the average transaction.
Interestingly, users do not seem to use these additional funds to build up
emergency savings: net-inflows into savings accounts do not change after
signup. I can also neither find significant increases in flows into investment
and pension accounts or additional savings accounts that are not linked to the
app, not additional loan repayments.

My work mainly contributes to three strands of the literature. The first, is
the aforementioned recent literature that studies the effect of FinTech apps on
financial outcomes. The second, is the very recent literature on studying
interventions to help increase emergency savings an area of household finances
that has until recently had no attention. In addition to studies testing the
effect of FinTech apps on savings, there is a strand of research that studies
the use of auto-enrolment into employer-sponsored savings accounts -- similar
to the ones used to increase pension savings \citep{thaler2004save,
choi2004better, choukhmane2019default} -- and finds an increase in both
participation (relative to opt-in accounts) and account balances
\citep{beshears2020building, berk2022automating}. Finally, my work also
contributes to a rapidly growing literature of using financial-transaction data
from banks or financial aggregator apps to understand consumer financial
behaviour. As already mentioned, \citet{kuchler2020sticking} use data from a
financial aggregator app to estimate time preferences. Similar data has been
used to show that consumer spending varies across the pay cycle
\citep{gelman2014harnessing,olafsson2018liquid}, to test the consumer spending
response to exogenous shocks \citep{baker2018debt,baugh2014disentangling}, and
to better understand the generational differences in financial platform usage
patterns \citep{carlin2019generational}. Some researchers use transaction-data
directly provided by banks. \citet{ganong2019consumer} show that consumer
spending drops sharply after the predictable income drop from exhausting
unemployment insurance benefits, \citet{meyer2018fully} analyse how individuals
reinvest realised capital gains and losses, and \citet{muggleton2020evidence}
show that chaotic spending behaviour is a harbinger of financial
distress.\footnote{For a comprehensive review of the literature using financial
transaction data, see \citet{baker2022household}.}

There are two main limitations to my approach. First, the data is not
generated by a randomised experiment. The gold-standard to evaluate whether use
of Money Dashboard improves financial outcomes would be a randomised
controlled-trial, where out of a sample of potential users (ideally random and
representative of the UK population), we would randomly grant access to the app
to some users and then compare outcomes of those treated users with the control
group of users who did not have access. Instead, the data I have access to
only contains data for individuals who self-selected into using the app.
Individuals will choose to do so for a number of different reasons, all of
which are unobserved in the data, and at least some of which would probably
have changed their financial outcomes even if they had not signed up to the
app. Any changes in financial outcomes we observe are thus ``aggregate'' or
``net'' effects of these unobservables and the ``pure'' causal effect of app
use.

To see this, think of the net effect as $\textit{net effect} = \textit{causal
effect} + \textit{``need''}(\downarrow) + \textit{``motivation''}(\uparrow)$,
where the arrows indicate the direction of the bias, and consider three cases
that illustrate three stylised but plausible scenarios for signup. First,
consider a user who signs up in the hope that the app will help them reign in
discretionary spending that has has gotten out of hand. If it takes the user
some time to fully adjust their spending, then even if the app does help them
make these adjustments, the estimated positive effect of app use will be biased
downward. Next, consider a user who decides to start bringing their own lunch
to work instead of eating out in an effort to save for a new car and signs up
to MDB in the hope that the app will help them keep track of their spending.
Such a user would probably have reduced their discretionary spend even if they
had not signed up to the app, thus creating an upward bias on our estimated net
effect. Finally, consider a user who signs up to MDB purely because they
happened to see an advert for the app on the Bus and got curious. In this case,
we can think of signup being close to random -- almost as if the user had been
allocated to the treatment group in our ideal experiment -- and the estimated
net effect will closely resemble the causal effect of app use. Hence, under the
weak assumption that at least some users sign up for reasons that are not as
good as random, our estimated effects will be biased upwards or downwards
depending on the relative proportion of users whose unobservable reasons for
signup create an upward and downward bias.

The second limitation is that even if I were able to isolate the effect of
the app, I am not able to differentiate between the contributions of different
features of the app such as improved access to information and budgeting.

Despite these limitations, which mean that we cannot interpret results as
causal effects of using Money Dashboard, the finding that app use is associated
with an economically significant and sustained drop in discretionary spend, and
the finding that this reduction comes about by users making fewer transactions
in a number of different spending categories month-by-month provides
interesting insight of how users reduce discretionary spend, either because of
or simply while using Money Dashboard. Furthermore, the results make it very
plausible that the Money Dashboard, and possibly apps like it, does have a
positive causal effect, and thus suggests that further research, focused on
identifying the causal effect of the app overall as well as that of its
component parts, would be interesting and worthwhile.

The remainder of this paper is organised as follows: Section~\ref{sec:data}
introduces the dataset used, discusses preprocessing and presents summary
statistics; Section~\ref{sec:estimation} introduces the empirical approach used
in the analysis; Section~\ref{sec:results} presents the results; and
Section~\ref{sec:conclusion} concludes. To make it easier for interested
readers to clarify questions about details and subtleties of data preprocessing
and analysis steps, I provide links to the scripts that implement the steps
discussed in the text in the relevant places throughout the text.\footnote{The
    projects GitHub repo that contains all files used to produce the results
    can be found at
\href{https://github.com/fabiangunzinger/mdb\_eval}{https://github.com/fabiangunzinger/mdb\_eval}.}

