% !TEX root = ../eval.tex

\section{Introduction}%
\label{sec:introduction}

This paper evaluates whether using Money Dashboard, a UK-based financial
aggregator app, is associated with a reduction in discretionary spend and an
increase in emergency savings -- short-term savings intended as a buffer
against unexpected financial shocks.

The question is important because a large number of adults in the UK and the US
do not have enough savings to cover unexpected expenses like car or medical
bills: in the UK, 25 percent of adults would be unable to cover an unexpected
bill of \pounds300 \citep{phillips2021supporting}, while in the US, about 30
percent would be unable to cover a \$400 bill \citep{fed2022economic}.

A large literature has documented a variety of factors that can influence
undersaving and financial decision-making more generally: time-preferences and
present bias \citep{laibson1997golden, frederick2002time,
read2018intertemporal, ericson2019intertemporal, cohen2020measuring}, inertia
\citep{madrian2001power}, over-extrapolation \citep{choi2009reinforcement},
limited self-control and willpower \citep{thaler1981economic,
  benhabib2005modeling, o1999doing, fudenberg2006dual, loewenstein2004animal,
gul2001temptation}, cognitive limitations and financial literacy
\citep{agarwal2009age, agarwal2013cognitive, korniotis2011older,
agarwal2010learning, fernandes2014financial, jorring2020financial}, attitude
towards money and spending \citep{rick2008tightwads, rick2011fatal}, one's
perceived locus of control \citep{perry2005control}, degree of optimism
\citep{puri2007optimism}, the ability to frame decisions broadly rather than
narrowly \citep{kumar2008decision}, propensity to gamble
\citep{kumar2009gambles}, one's social network \citep{bailey2018economic,
kuchler2021social}, the degree of one's financial planning
\citep{ameriks2003wealth}, and habits \citep{blumenstock2018defaults,
schaner2018persistent, de2013deposit}.\footnote{For two thorough reviews of the
literature, see \citet{agarwal2017shapes} and \citet{greenberg2019financial}.}

Researchers have also identified a number of tools that can help people
overcome some of these biases. Probably the most studied approach is changing
default options. Changes in defaults are consequential and often successful
because people often stick with the status quo \citep{samuelson1988status},
interpret defaults as a recommendation \citep{mckenzie2006recommendations},
implicitly view them as reference points they do not want to move away from
\citep{johnson2003defaults, kahneman1979prospect}, and often end up sticking
with them even if they intend to to otherwise because of procrastination
\citep{carroll2009optimal, ericson2017interaction}. Default options have been
applied across range of areas and have, for instance, been found to increase
retirement savings contributions \citep{thaler2004save, madrian2001power,
beshears2009importance} and organ donations \citep{johnson2003defaults,
gimbel2003presumed, abadie2006impact}.

Another extensively studied approach is the commitment device, whereby an
individual restricts their future choice set in order to avoid choosing a
self-defeating action. Not everyone makes use of these devices when they are
offered \citep{bryan2010commitment} and they do not work in all contexts
\citep{laibson2015don,robinson2018some}, but they have been found to help
individuals increase their savings rates \citep{ashraf2006tying}, quit smoking
\citep{gine2010put}, make healthier food choices \citep{schwartz2014healthier},
and exercise more regularly \citep{royer2015incentives}.

A third approach are implementation intentions, a particular type of planning
for the achievement of one's goals that involves ``if-then'' intentions, such
as ``if I get paid, then I transfer 10 percent of it into my savings account''
\citep{gollwitzer2006implementation, rogers2015beyond}. Such intentions have
been found to support perseverance in pursuing one's goals
\citep{oettingen2010strategies} and to increase overall goal attainment across
different age groups, life domains, and types of obstacles
\citep{gollwitzer2006implementation}.

Finally, social norms messaging, whereby people are informed about how their
own behaviour compares to that of a relevant peer group, has been successful in
inducing high-energy-use households to lower their energy consumption
\citep{schultz2007constructive, allcott2011social, allcott2014short,
brandon2017effects}. Such information can be especially useful for domains
where people usually are uninformed about the behaviour of their peers and do
thus not have a reference point against which to calibrate their own behaviour
-- a situation most people find themselves in when it comes to discretionary
spending and saving. However, social norms messaging can also backfire and has
been found to lower participation in pension savings plans
\citep{beshears2015effect} and completion rates of an online course
\citep{rogers2016discouraged}, making it important to test messages before
deploying them on a large scale to avoid unintended consequences.

FinTech aggregator apps such as Money Dashboard are a promising mean to help
people manage their spending and savings for two reasons: eventually, as these
apps become more sophisticated, they can be used to implement and test the
above mentioned tools at a large scale, and thus potentially bring their
benefit from research labs or specifically designed tools used by a small
minority to the broader population. In the meantime, they provide additional
features that -- on their own -- can help people manage their finances more
successfully. In particular, they provide easy access to financial information
that makes it easier to monitor one's spending and saving, and often also offer
tools such as budgeting and the setting of spending goals. Financial
information that is more easily accessible and is aggregated in ways that help
people keep track of their goals might be beneficial because rational
inattention theory predicts that it makes people more likely to access that
information, which, in turn, might lead to better consumption decisions
\citet{brunnermeier2008wealth, dellavigna2009psychology, sims2003implications}.
Tools that help with budgeting and with setting spending goals have the
potential to help users make consumption decisions more in line with their
intentions because such tools can act as commitment devices that -- if users
experience disutility from falling short of their goals -- introduce a
cognitive cost to overspending or undersaving.

In this paper, I specifically test whether using Money Dashboard -- an
aggregator app that offers aggregation of all financial accounts and budgeting
features -- is associated with a reduction in discretionary spending and an
increase in emergency savings. I use a new estimator proposed by
\citet{callaway2021difference} that corrects for recently identified problems
in two-way fixed effects estimates.

I find that users reduce their discretionary spend by between \pounds100 and
\pounds150 per month (11-17\% of average discretionary spend) once they start
using the app and sustain that reduction throughout the six-month post-signup
period I consider. Looking at disaggregated measures of discretionary spend
further shows that the reduction is the result of maintained month-to-month
changes in behaviour rather than one-off cancellations of direct-debit
transactions, that it results from reducing spending on a number of different
categories of purchases rather than a single one, and that it is a result of
changes along the extensive rather than the intensive margin -- users reduce
the number of transactions they make rather than the value of the average
transaction.  Interestingly, users do not seem to use these additional funds to
build up emergency savings: net-inflows into savings accounts do not change
after signup. I can also neither find significant increases in flows into
investment and pension accounts or additional savings accounts that are not
linked to the app, not additional loan repayments. This suggests that users
either leave unspent funds to accrue in their current accounts (which would not
count to my definition of savings), of allocate them across a number of
different uses in small amounts I am not powered to detect.

There are two main limitations to my approach. First, the data is not
generated by a randomised experiment. The gold-standard to evaluate whether use
of Money Dashboard improves financial outcomes would be a randomised
controlled-trial, where out of a sample of potential users (ideally random and
representative of the UK population), we would randomly grant access to the app
to some users and then compare outcomes of those treated users with the control
group of users who did not have access. Instead, the data I have access to
only contains data for individuals who self-selected into using the app.
Individuals will choose to do so for a number of different reasons, all of
which are unobserved in the data, and at least some of which would probably
have changed their financial outcomes even if they had not signed up to the
app. Any changes in financial outcomes we observe are thus ``aggregate'' or
``net'' effects of these unobservables and the ``pure'' causal effect of app
use.

To see this, think of the net effect as $\textit{net effect} = \textit{causal
effect} + \textit{``need''}(\downarrow) + \textit{``motivation''}(\uparrow)$,
where the arrows indicate the direction of the bias, and consider three cases
that illustrate three stylised but plausible scenarios for signup. First,
consider a user who signs up in the hope that the app will help them reign in
discretionary spending that has has gotten out of hand. If it takes the user
some time to fully adjust their spending, then even if the app does help them
make these adjustments, the estimated positive effect of app use will be biased
downward. Next, consider a user who decides to start bringing their own lunch
to work instead of eating out in an effort to save for a new car and signs up
to MDB in the hope that the app will help them keep track of their spending.
Such a user would probably have reduced their discretionary spend even if they
had not signed up to the app, thus creating an upward bias on our estimated net
effect. Finally, consider a user who signs up to MDB purely because they
happened to see an advert for the app on the Bus and got curious. In this case,
we can think of signup being close to random -- almost as if the user had been
allocated to the treatment group in our ideal experiment -- and the estimated
net effect will closely resemble the causal effect of app use. Hence, under the
weak assumption that at least some users sign up for reasons that are not as
good as random, our estimated effects will be biased upwards or downwards
depending on the relative proportion of users whose unobservable reasons for
signup create an upward and downward bias.

The second limitation is that even if I were able to isolate the effect of
the app, I am not able to differentiate between the contributions of different
features of the app such as improved access to information and budgeting.

Despite these limitations, which mean that we cannot interpret results as
causal effects of using Money Dashboard, the finding that app use is associated
with an economically significant and sustained drop in discretionary spend, and
the finding that this reduction comes about by users making fewer transactions
in a number of different spending categories month-by-month provides
interesting insight of how users reduce discretionary spend, either because of
or simply while using Money Dashboard. Furthermore, the results make it very
plausible that Money Dashboard, and possibly apps like it, does have a
positive causal effect, and thus suggests that further research, focused on
identifying the causal effect of the app overall as well as that of its
component parts, would be interesting and worthwhile.

My work mainly contributes to three strands of the literature. The first is the
nascent literature that studies the effect of FinTech apps on financial
outcomes suggests. This literature suggests that these apps can indeed lead to
improved financial outcomes through a number of different channels: providing
users with information about their spending relative to peers has been found to
reduce spending \citep{dacunto2020crowdsourcing}; offering budgeting options,
to reduce spending \citep{lukas2022influence}; goal setting, to increase
savings \citep{gargano2021goal}; and facilitating access to financial
information, to reduce non-sufficient funds fees \citep{carlin2022mobile} and
discretionary spend \citep{levi2020mind}.

The two studies that are most closely related to my work are
\citet{lukas2022influence} and \citet{gargano2021goal}.
\citet{lukas2022influence} also use data from Money Dashboard to study the
effect of budgeting on discretionary spend. In line with my finding, they find
that budgeting reduces discretionary spend in the associated categories and
that this effect persists throughout the six-month post-budgeting period they
study. Their approach also suffers from the two main limitations of my paper
discussed above: budgeting is not randomly assigned, and they cannot isolate
the effect of budgeting from other MDB features that might help users reduce
their spending. There are two main differences to my work: first,
\citet{lukas2022influence} do not study the effect of MDB use on savings.
Second, they use data from between January 2014 and December 2016, and do not
select their sample to ensure that it only contains users for whom they can
observe all financial accounts. As I discuss in
Section~\ref{sub:preprocessing}, there is a large number of users in the raw
dataset for whom we are unlikely to observe all transactions, making careful
sample selection critical to ensure that observed changes in behaviour do not
merely reflect a shift of transactions between observed and unobserved
accounts. \citet{gargano2021goal} study the effect goal setting on savings and
find that setting savings goals does increase savings rates. Their approach
does not suffer from the two limitations of my paper since they exploit the
random assignment of some users into a group of beta testers that can set
savings goals, and because the savings app that provides the data for their
study was initially designed as a simple savings app and provides no additional
features that could plausibly influence savings behaviour. However, the
flip-side of being able to cleanly identify the effect of goal setting is that
most apps that people use to increase their savings or reduce their spending do
-- like Money Dashboard -- have multiple features that might alter financial
behaviour, so that studying their joint effect is of interest, too.

The second strand of related literature is the very recent literature on studying
interventions to help increase emergency savings an area of household finances
that has until recently had no attention. In addition to studies testing the
effect of FinTech apps on savings, there is a strand of research that studies
the use of auto-enrolment into employer-sponsored savings accounts -- similar
to the ones used to increase pension savings \citep{thaler2004save,
choi2004better, choukhmane2019default} -- and finds an increase in both
participation (relative to opt-in accounts) and account balances
\citep{beshears2020building, berk2022automating}.

Finally, my work also contributes to a rapidly growing literature of using
financial-transaction data from banks or financial aggregator apps to
understand consumer financial behaviour. As already mentioned,
\citet{kuchler2020sticking} use data from a financial aggregator app to
estimate time preferences. Similar data has been used to show that consumer
spending varies across the pay cycle
\citep{gelman2014harnessing,olafsson2018liquid}, to test the consumer spending
response to exogenous shocks \citep{baker2018debt,baugh2014disentangling}, and
to better understand the generational differences in financial platform usage
patterns \citep{carlin2019generational}. Some researchers use transaction-data
directly provided by banks. \citet{ganong2019consumer} show that consumer
spending drops sharply after the predictable income drop from exhausting
unemployment insurance benefits, \citet{meyer2018fully} analyse how individuals
reinvest realised capital gains and losses, and \citet{muggleton2020evidence}
show that chaotic spending behaviour is a harbinger of financial
distress.\footnote{For a comprehensive review of the literature using financial
transaction data, see \citet{baker2022household}.}


The remainder of this paper is organised as follows: Section~\ref{sec:data}
introduces the dataset used, discusses preprocessing and presents summary
statistics; Section~\ref{sec:estimation} introduces the empirical approach used
in the analysis; Section~\ref{sec:results} presents the results; and
Section~\ref{sec:conclusion} concludes. To make it easier for interested
readers to clarify questions about details and subtleties of data preprocessing
and analysis steps, I provide links to the scripts that implement the steps
discussed in the text in the relevant places throughout the text.\footnote{The
    projects GitHub repo that contains all files used to produce the results
    can be found at
\href{https://github.com/fabiangunzinger/mdb\_eval}{https://github.com/fabiangunzinger/mdb\_eval}.}

