% !TEX root = ../eval.tex

\section{Introduction}%
\label{sec:introduction}


% people undersave
\begin{itemize}
    \item Long-term savings: there is large literature on this (document
        problem and lit)

    \item Short-term savings: there is almost no literature on this.

    \item About 25 percent of adults in the UK would be unable to cover an
        unexpected bill of \pounds300 \citep{philipps2021supporting}, while
        about 30 percent of US adults would be unable to cover a bill of with
        their own money \citep{fed2022economic}.

    \item Behavioural biases for undersaving well documented...

    \item Remedies not so much...

\end{itemize}


Emergency savings:
\begin{itemize}

    \item Automation: \citet{beshears2020building}: explore different options
        of employer-sponsored accounts ``for rainy-day savings``.
        \citet{berk2022automating} evaluate initial data from an experiment run
        at a large UK-employer and find that defaulting employees into payroll
        savings accounts increases participation by 50 percent and balances by
        about \pounds68 compared to an opt-in scheme.

\end{itemize}

Possible channels of fintech benefits
\begin{itemize}
    
    \item Information: \citet{levi2020mind}
        \begin{itemize}

            \item Rational inattention theory

            \item Pay more attention to finances \citet{stango2014limited,
                medina2021side, bursztyn2019moral}

            \item Make fewer mistakes \citet{stango2009consumers,
                jorring2020financial}.
        \end{itemize}

\end{itemize}

Benefits of fintech apps:
\begin{itemize}

    \item Peer information: \citet{dacunto2020crowdsourcing} show that peer
        information might help (similar to energy use studies
        \citet{allcott2014short}).

    \item Budgetting/goal setting: \citet{gargano2021goal, lukas2022influence}

    \item Reduced non-sufficient funds fees \citet{carlin2022mobile}

\end{itemize}

% financial transaction data literature
Finally, my work also contributes to a rapidly growing literature of using
financial-transaction data from banks or financial aggregator apps to understand
consumer financial behaviour. As already mentioned, \citet{kuchler2020sticking}
use data from a financial aggregator app to estimate time preferences. Similar
data has been used to show that consumer spending varies across the pay cycle
\citep{gelman2014harnessing,olafsson2018liquid}, to test the consumer spending
response to exogenous shocks \citep{baker2018debt,baugh2014disentangling}, and
to better understand the generational differences in financial platform usage
patterns \citep{carlin2019generational}. Some researchers use transaction-data
directly provided by banks. \citet{ganong2019consumer} show that consumer
spending drops sharply after the predictable income drop from exhausting
unemployment insurance benefits, \citet{meyer2018fully} analyse how individuals
reinvest realised capital gains and losses, and
\citet{muggleton2020evidence} show that chaotic spending behaviour is a
harbinger of financial distress.

