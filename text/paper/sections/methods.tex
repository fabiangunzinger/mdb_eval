% !TEX root = ../eval.tex

\section{Methods}%
\label{sec:data}

While we were unable to pre-register the analysis because we have had access to
and been working with the Money Dashoard data for months, we proceeded in the
same spirit: we first wrote a draft of the paper in the form of a pre-analysis
plan, following \citet{olken2015promises}, then tested the entire code base --
data pre-processing, balance checks, main analysis, and extensions -- with a 1
percent sample, and finally ran the entire analysis.


\subsection{Dataset}%
\label{sub:dataset}

Limitations:
\begin{itemize}
    \item We have more data for users that signed up later. So average user in
        the study is not the average MDB user. If time of signup is mainly
        driven by financial savyness, then study sample is closer to overall
        population than MDB sample (if we rank groups as early joiners > late
        joiners > never joiners in terms of financial sophistication). If,
        however, signup reflects something like openness to newness, then it's
        not necessarily correlated with financial savyness. Either way, we
        might ignore it for now. We could test whether behaviour differs
        between early or late adopters, but that doesn't seem important enough.
\end{itemize}


\subsection{Sample selection}%
\label{sub:sample_selection}

To assess the impact of MDB use on users' financial behaviour we need to
observe their relevant financial history for a sufficiently long period of time
prior to and after they started using the app. For our purpose here, ``relevant
financial history'' includes the complete set of spending transactions and all
savings account inflows and outflows, and ``sufficiently long period'' is a
period of 6 months prior to and after signup, with the month of signup being
the first month of the latter period.\footnote{In
Appendix~\ref{sub:alternative_matching_method} we show results with different
window lengths. \edit{The results are unchanged.}}

Table~\ref{tab:selection} provides an overview of the precise conditions we applied to implement
these criteria and their effect on the sample size. The set of functions that
implement each condition can be found on \href{path-to-github}{path to github}.

\begin{table}
\centering
\caption{Sample selection}\label{tab:selection}
\begin{tabular}{lrrrr}
\toprule
                                       &  Users & User-months &       Txns & Txns (m\pounds) \\
\midrule
                            Raw sample & 27,329 &     803,012 & 67,088,876 &          12,722 \\
Annual income of at least \pounds5,000 &  9,338 &     256,891 & 24,582,995 &           4,463 \\
          At least one savings account &  5,579 &     167,705 & 16,952,445 &           3,249 \\
            At least 12 months of data &  4,884 &     163,602 & 16,657,929 &           3,207 \\
  Monthly spend of at least \pounds200 &  2,506 &      82,860 &  9,303,691 &           1,823 \\
                          Final sample &  2,506 &      82,860 &  9,303,691 &           1,823 \\
\bottomrule
\end{tabular}

\end{table}


\subsection{Treatment}%
\label{sub:treatment}

Figure~\ref{fig:treatplot_sample_raw} shows treatment assignment for a raw
sample of random subset of 200 users. Compare to treatplot of final sample and
talk about important implications...

\begin{figure}[htpb]
    \centering
    \caption{Treatment assignment plot for raw sample}%
    \includegraphics[width=0.8\linewidth]{\figdir/treatplot_sample_raw.png}
    \label{fig:treatplot_sample_raw}
\end{figure}
\fignote{\textwidth}{Screenshot from the top of the Money Dashboard website, at
\href{https://www.moneydashboard.com}{moneydashboard.com}, accessed on 29 April
2022.}


\subsection{Outcomes}%
\label{sub:outcomes}

\begin{table}[htpb]
    \centering\footnotesize
    \caption{Outcome description}
    \label{tab:label}
    \begin{tabularx}{\textwidth}{>{\raggedright\arraybackslash}X
        >{\raggedright\arraybackslash}X>{\raggedright\arraybackslash}X}
    \hline\hline
    Outcome (name in dataset) & Definition  & Rationale \\
    \hline\\
    \multicolumn{3}{c}{Primary outcome}\\\\

    Net savings (\textit{sa\_netflows\_norm})&
    Inflows into minus outflows out of all of a user's savings accounts divided
    by monthly income. To capture only ``user-generated'' flows, we exclude
    interest and ``save the change'' transactions, as well as transactions of
    less than \pounds5 in absolute
    value. Monthly income and raw inflows and outflows are winsorised at the 1
    percent level.\href{https://github.com/fabiangunzinger/mdb_eval/blob/d094f8cd364f64bbe3d4e644abbff726af86de2f/src/data/aggregators.py\#L27}{\faGithub}&
    We focus on net inflows to capture effective savings.\\\\

    \multicolumn{3}{c}{Secondary outcomes}\\\\

    Positive net savings dummy (\textit{has\_pos\_sa\_netflows})&
    Dummy equal to 1 if there were positive net savings (as defined above).&
    Captures extensive margin of savings (change in number of deposits).\\

    Positive net savings (\textit{pos\_sa\_netflows})&
    Equal to net savings if there were positive net savings.&
    Captures intensive margin of savings (change in deposit amount).\\

    ...add from text below&&\\

    \hline\hline
    \end{tabularx}
\end{table}

For a more nuanced understanding of how app use affects savings we also
consider net-savings -- total savings account inflows minus outflows -- as a
proportion of monthly income to see whether a willingness to save more might be
offset by a (later) need to withdraw funds, and a dummy variable for whether a
user has any savings account inflows in a given month to see whether the app
helps users save at all. To investigate possible channels, we consider total
spend, highly discretionary spend, banking charges, the total amount of
borrowing, as well as payday borrowing, all as proportion of monthly income.

\paragraph{Adjusting for multiple hypothesis tests}%
\label{par:adjusting_for_multiple_hypothesis_tests}
We think of our secondary outcomes as exploratory and do not make any
adjustments for multiple hypothesis testing.\footnote{For a recent
game-theoretically motivated discussion of when and how to correct for multiple
hypothesis testing, see \citet{viviano2021should}.} An alternative approach,
based on \citet{anderson2008multiple}, would be to group outcomes into
``savings'', ``spending'', ``borrowing'', and ``fees'', and consider them as
different dimensions of a latent variable of interest which we might call
``financial management skills''. We do not do that for two reasons: first and
foremost, because we think it is natural to think of the amount saved as the
ultimate outcome and of other outcomes as providing a more nuanced
understanding of savings behaviour or as suggesting possible channels through
which app use affects savings. Thinking of savings as the main goal is also
reflected in Money Dashboard's main promise, which is to help users spend less
and save more, as shown in Figure~\ref{fig:mdb_website}. Second, as pointed out
in \citet{carlin2017fintech}, incurring overdraft fees is not an unambiguous
sign of a financial mistake, as the opportunity to go into overdraft confers a
benefit to the consumer.\footnote{For further discussions on fees, see
\citet{jorring2020financial, stango2009consumers}.}


\subsection{Covariates}%
\label{sub:covariates}

Description of covariates.


\subsection{Difference-in-difference}%
\label{sub:difference_in_difference}

Control group design:
\begin{itemize}

    \item We only have data for a self-selected group of people who choose to
        use Money Dashboard. By virtue of signing up to an app that helps them
        manage their money, these users are different from those who don't sign
        up. As a result, we are unable to answer the question of whether use of
        Money Dashboard helps the average person in the population as a whole
        save more.\footnote{One way to get closer to that answer is to
            re-weight our sample on observable demographic variables so as to
            match the UK population as a whole. But our sample differs from the
            population as a whole both is ways that are observable (demographic
            variables) and unobservable (self-awareness that they need help
            managing their money, cognitive resources to engage with the app,
            motivation to do so). Re-weighting would only help us deal with the
        first of these.} Instead, we are answering the question whether Money
        Dashboard succeeds in helping its \textit{users} save more.

    \item Money Dashboard can access up to three years of historic data for
        each account a user links to their account.

    \item Each user for whom we have sufficient data thus serves as both a
        treatment unit and a potential control unit.

    \item We use a difference-in-differences design to estimate the effect of
        app use. Because we do not have a separate control group, we use the
        per-signup data of Money Dashboard users as control periods and use
        matching to find comparable control user for each tretment user.

    \item To do this, we use the matching estimator for panel data proposed by
        \citet{imai2021matching}. Following paper, we conduct the following
        steps:

    \item For each treated observation, we find a set of control observations
        with that share the same treatment history for a period of $L$ periods
        before the treatment and $F$ periods after the treatment. In our
        baseline specification, we rely on a year's worth of data around the
        treatment period and set $L=6$ and $F = 0, 1, 2, 3, 4, 5$.

    \item Identification assumption is that potential outcomes only depend on
        treatment status of the past L periods. In general, this means that if
        treatment has a cumulative effect over time, the full effect is reached
        after L periods. In our context, this means that any effect on savings
        behaviour from usign the app is fully realised after L periods. (I
        think this means that if we look at the treatment effect for F periods
        forward, the effect should not become stronger after F = L).

    \item tbc once implemented.


    \item Selection of covariates: all variables that simultaneously affect
        treatment and outcomes. No need to control for fixed effects: these
        capture unobserved time-invariant factors that make an individual sign
        up to MDB and affect its spending habits. Given that these are time
        invariant, and that all users eventually sign up, there is no
        difference between control and treatment units in these factors. Month
        of year: should probably include, as can affect p of signup and
        spending behaviour.

\end{itemize}





Is estimate causal?
\begin{itemize}
    \item \citet{king2006dangers} show that there are four sources of bias
        (ommitted variable, posttreatment, interpolation, extrapolation).
    
    \item Discuss each in turn to argue that effect is causal (for our population
        of interest, which are people signing up to MDB). 
\end{itemize}


