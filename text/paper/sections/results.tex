% !TEX root = ../eval.tex

\section{Results}%
\label{sec:results}


\subsection{Dynamic TWFE}%
\label{sub:dynamic_results}

\begin{equation}
\label{eq:dynamic_twfe}
    y_{it} = \sum^{5}_{s=-6} \beta_s D_{it}^s + \alpha_i + \lambda_t + \gamma X_{it} + \epsilon_{it},
\end{equation}

where $y_{it}$ is the outcome for individual $i$ at time $t$, $\alpha_i$ and
$\lambda_t$ are individual and year-month fixed effects, respectively, and
$X_{it}$ is a vector of individual and time varying controls. $D_{it}^s$ equals
1 if, in period $t$, individual $i$ is $s$ months away from signing up to the
app. The set of $\beta_s$ coefficients measure the effect of treatment $s$
periods away from treatment, which is what we are interested in.

To avoid multicollinearity issues and uniquely identify the lead and lag
effects, we omit relative period indicator $s = -1$, and bin the endpoints of our
observation window so that $D_{it}^{-6} = 1$ for all periods six or more
periods prior to signup and $D_{it}^{5} = 1$ for all months five or more months
after the month of signup. This expresses all estimates relative to the omitted
period, which -- given that it is the last month before signup -- serves as a
natural benchmark. Binning endpoints assumes that coefficients remain constant
outside the observation window. Both of these approaches are standard in the
literature, as discussed in \citet{sun2021estimating, schmidheiny2019event}.

% We omit the relative period indicator for period $s = -1$ because we need to
% omit one relative period indicator to avoid perfect collinearity among the
% period indicators, and we choose the last pre-treatment period because it
% serves as a natural benchmark against which to compare the outcomes in other
% periods.\footnote{As \citet{sun2021estimating} point out, there are two sources
% of perfect multicollinearity when estimating a fully dynamic model (i.e. one
% including all possible lags). The first results from all relative period
% indicators summing to 1 in each period, so that the entire set of relative period
% dummies across all time periods is perfectly multicollinear. We deal with this
% by excluding the indicator for $s = -1$. The second issue arises from the fact
% that for initial treatment period $E_i$, $t = s + E_i$. We deal with this issue
% by "trimming" our sample to be balanced in relative periods by only using
% data from relative periodl $\{-6, 5\}$. Both of these approaches are standard
% in the empirical literature.}

\begin{figure}[H]
    \centering
    \caption{Main results}%
    \label{fig:dspend_main}
    \includegraphics[width=.49\textwidth]{\figdir/dspend_main.png}
    \includegraphics[width=.49\textwidth]{\figdir/netflows_main.png}
    \fignote{\textwidth}{Figures show estimates of leads and lags frome
        estimating model~(\ref{eq:dynamic_twfe}) with distretionary spend (left) and
netflows into savings accounts (right) as the dependent variable. Standard
errors are clustered by user.}
\end{figure}





% \citet{sun2021estimating} define an event study design as a staggered adoption
% design where units are treated at different times and where there may or may
% not be never treated units. In our case, we have no never treated units, and
% treatment is absorbing in that once a unit is treated they will also be treated
% in all subsequent units.\footnote{We cannot rule out that some users who
% stopped using the app and closed their account rejoined later on, in which case
% they would appear in our dataset as a new user. However, we can plausibly
% assume that such cases are rare.} 

% Setup:
% \begin{itemize}
%     \item We observe $N+1$ unites for $T+1$ periods and, for each
%         $i\in\{0,\ldots, N\}$ and $t\in\{0,\ldots,T\}$ observe outcome $y_{it}$
%         and treatment status $D_{it}\in\{0, 1\}$, where $D_{it}$ equals 1 if unit
%         $i$ is treated in period $t$ and 0 otherwise.

%     \item We can uniquely characterise treatment paths by the time period of
%         initial treatment, denoted as $E_i = min\{t: D_{it} = 1\}$.

%     \item We can group units into cohorts $e \in \{0,\ldots, T\}$, where units
%         in cohort $e$ were all first treated at time $e$, so that $\{i: E_i =
%         e\}$.

%     \item We define $y^e_{it}$ as the potential outcome in period $t$ if unit
%         $i$ was first treated in period $e$.

% \end{itemize}


% Assumptions:
% \begin{itemize}
%     \item Observations $\{y_{it}, D_{it}\}_{t=0}^T$ are independent.

%     \item A1: parallel trends: difference in baseline outcomes over time do
%         not differ between treatment cohorts. Not obviously violated in our
%         context. Early adopters might differ from late adopters, but difference
%         might plausibly be constant over time. - We don't have never treated
%         units, so Ashenfelter dip scenario is not a problem, even though we
%         seem to observe something like this in discretionary spending graph
%         (increase in disc spend before signup)

%     \item A2: no anticipatory behaviour. Plausibly violated if people are
%         motivated to save more and start doing so even before app use. Can test
%         for whether there is a peak prior to signup. Because our units have
%         private knowledge about future of treatment path (their intention to
%         reduce spending and save more and sign up to an app), this might be
%         violated. The trajectories of discret spend and net inflows are
%         conflicting on this, though, suggesting that they increases discret
%         spend in runup to app use (which might provide motivation to eventually
%         sign up) but also might have increased net savings slightly.

%     \item A3: treatment effect homogeneity across all cohorts and all relative
%         periods. (Note: treatment effects can be dynamic, but need to be the
%         same across cohorts). We could test for this.

% \end{itemize}


% Notes:
% \begin{itemize}

%     \item Comparison: pre vs post signup within each individual.

%     \item Assumption: there are no time-varying unobserved effects that affect
%         both y and D (formally: $E[Du] = 0$, since u is by definition
%         correlated with y.

%     \item Discussion: there is something that made the individual sign up in
%         the first place, and it might well be an individual level shock that we
%         don't observe (unexpected large expense, loss of job, exposure to
%         something that motivates saving or change in financial behaviour).

%     \item See \citet{imai2021use} for problems with twfe

% \end{itemize}



\subsection{Decomposing intensive and extensive margins}%
\label{sub:decomposing_intensive_and_extensive_margins}

\begin{figure}[H]
    \centering
    \caption{Decomposition of intensive and extensive margins}%
    \label{fig:intext}
    \includegraphics[width=.49\textwidth]{\figdir/has_pos_netflows.png}
    \includegraphics[width=.49\textwidth]{\figdir/pos_netflows.png}
    \includegraphics[width=.49\textwidth]{\figdir/dspend_count.png}
    \includegraphics[width=.49\textwidth]{\figdir/dspend_mean.png}
    \fignote{\textwidth}{...}
\end{figure}


% 
\begin{table}[htbp]
   \centering
   \tiny
   \begin{threeparttable}[b]
      \caption{\label{tab:intext} Intensive and extensive margins}
      \begin{tabular}{lcccc}
         \tabularnewline \midrule \midrule
         Dependent Variables:              & Has positive net-inflows & Positive net-inflows & Discretionary spend txns & Mean discretionary spend txn\\  
         Model:                            & (1)                      & (2)                  & (3)                      & (4)\\  
         \midrule
         \emph{Variables}\\
         Months relative to app use $=$ -6 & 0.02$^{***}$             & 68.33$^{**}$         & -2.98$^{***}$            & 1.81$^{***}$\\   
                                           & [0.01; 0.04]             & [0.10; 136.56]       & [-3.73; -2.24]           & [0.75; 2.87]\\   
         Months relative to app use $=$ -5 & 0.02$^{***}$             & 51.29                & -2.13$^{***}$            & 0.82$^{**}$\\   
                                           & [0.00; 0.03]             & [-26.47; 129.04]     & [-2.75; -1.52]           & [0.09; 1.54]\\   
         Months relative to app use $=$ -4 & 0.01                     & 1.77                 & -1.53$^{***}$            & 0.83$^{**}$\\   
                                           & [-0.01; 0.02]            & [-74.62; 78.15]      & [-2.08; -0.97]           & [0.14; 1.51]\\   
         Months relative to app use $=$ -3 & -0.01                    & -47.23               & -0.73$^{***}$            & 0.74$^{**}$\\   
                                           & [-0.02; 0.00]            & [-124.87; 30.40]     & [-1.20; -0.25]           & [0.01; 1.47]\\   
         Months relative to app use $=$ -2 & -0.01                    & -69.51$^{*}$         & -0.29                    & 0.58\\   
                                           & [-0.02; 0.00]            & [-141.13; 2.11]      & [-0.69; 0.10]            & [-0.19; 1.35]\\   
         Months relative to app use $=$ 0  & 0.00                     & 13.75                & -1.13$^{***}$            & -0.30\\   
                                           & [-0.01; 0.02]            & [-65.21; 92.72]      & [-1.54; -0.71]           & [-0.98; 0.37]\\   
         Months relative to app use $=$ 1  & 0.01                     & -21.45               & -1.54$^{***}$            & -0.11\\   
                                           & [-0.01; 0.02]            & [-96.59; 53.68]      & [-2.03; -1.05]           & [-0.82; 0.60]\\   
         Months relative to app use $=$ 2  & 0.01                     & -46.60               & -1.57$^{***}$            & 0.19\\   
                                           & [-0.00; 0.03]            & [-119.91; 26.70]     & [-2.13; -1.01]           & [-0.65; 1.03]\\   
         Months relative to app use $=$ 3  & 0.01$^{*}$               & -22.42               & -1.74$^{***}$            & 0.00\\   
                                           & [-0.00; 0.03]            & [-97.31; 52.48]      & [-2.37; -1.11]           & [-0.79; 0.79]\\   
         Months relative to app use $=$ 4  & 0.02$^{**}$              & -14.37               & -1.74$^{***}$            & 0.59\\   
                                           & [0.00; 0.03]             & [-89.61; 60.87]      & [-2.43; -1.04]           & [-0.58; 1.76]\\   
         Months relative to app use $=$ 5  & 0.02$^{**}$              & 28.69                & -2.96$^{***}$            & 0.93\\   
                                           & [0.00; 0.04]             & [-41.84; 99.21]      & [-3.88; -2.03]           & [-0.19; 2.05]\\   
         Month income                      & 0.00$^{***}$             & 0.06$^{***}$         & 0.00$^{***}$             & -0.00$^{***}$\\   
                                           & [0.00; 0.00]             & [0.04; 0.08]         & [0.00; 0.00]             & [-0.00; -0.00]\\   
         Month spend                       & -0.00$^{***}$            & 0.02$^{***}$         & 0.00$^{***}$             & 0.00$^{***}$\\   
                                           & [-0.00; -0.00]           & [0.01; 0.03]         & [0.00; 0.00]             & [0.00; 0.00]\\   
         Active accounts                   & 0.09$^{***}$             & 184.51$^{***}$       & 3.41$^{***}$             & -0.87$^{***}$\\   
                                           & [0.09; 0.10]             & [170.36; 198.66]     & [3.21; 3.62]             & [-1.21; -0.53]\\   
         \midrule
         \emph{Fixed-effects}\\
         User ID                           & Yes                      & Yes                  & Yes                      & Yes\\  
         Year-month                        & Yes                      & Yes                  & Yes                      & Yes\\  
         \midrule
         \emph{Fit statistics}\\
         Observations                      & 188,324                  & 188,324              & 188,324                  & 187,306\\  
         R$^2$                             & 0.28475                  & 0.11551              & 0.68049                  & 0.26390\\  
         Within R$^2$                      & 0.06323                  & 0.01277              & 0.17182                  & 0.01562\\  
         \midrule \midrule
         \multicolumn{5}{l}{\emph{Clustered (User ID) co-variance matrix, 95\% confidence intervals in brackets}}\\
         \multicolumn{5}{l}{\emph{Signif. Codes: ***: 0.01, **: 0.05, *: 0.1}}\\
      \end{tabular}
   \end{threeparttable}
\end{table}





\subsection{Decomposing inflows and outflows}%
\label{sub:decomposing_inflows_and_outflows}

\begin{figure}[H]
    \centering
    \caption{Decomposing inflows and outflows}%
    \label{fig:inout}
    \includegraphics[width=.6\textwidth]{\figdir/netflows.png}
    \includegraphics[width=.6\textwidth]{\figdir/inflows.png}
    \includegraphics[width=.6\textwidth]{\figdir/outflows.png}
    \fignote{\textwidth}{...}
\end{figure}


\begin{table}[htbp]
   \centering
   \tiny
   \begin{threeparttable}[b]
      \caption{\label{tab:inout} Decomposing inflows and outflows}
      \begin{tabular}{lccc}
         \tabularnewline \midrule \midrule
         Dependent Variables:              & Net-inflows       & Inflows          & Outflows\\  
         Model:                            & (1)               & (2)              & (3)\\  
         \midrule
         \emph{Variables}\\
         Months relative to app use $=$ -6 & -104.79$^{**}$    & 190.11$^{***}$   & 259.80$^{***}$\\   
                                           & [-197.45; -12.13] & [113.66; 266.56] & [187.96; 331.64]\\   
         Months relative to app use $=$ -5 & -45.90            & 144.54$^{***}$   & 162.75$^{***}$\\   
                                           & [-154.47; 62.67]  & [60.37; 228.70]  & [88.62; 236.88]\\   
         Months relative to app use $=$ -4 & -86.20            & 58.67            & 120.54$^{***}$\\   
                                           & [-195.61; 23.21]  & [-20.02; 137.37] & [51.23; 189.85]\\   
         Months relative to app use $=$ -3 & -92.58            & -43.36           & 23.62\\   
                                           & [-203.84; 18.67]  & [-120.62; 33.89] & [-44.16; 91.39]\\   
         Months relative to app use $=$ -2 & -119.53$^{**}$    & -37.28           & 51.52\\   
                                           & [-223.87; -15.19] & [-113.85; 39.30] & [-13.38; 116.42]\\   
         Months relative to app use $=$ 0  & -37.36            & 75.39$^{*}$      & 75.86$^{**}$\\   
                                           & [-150.33; 75.61]  & [-5.93; 156.72]  & [7.16; 144.56]\\   
         Months relative to app use $=$ 1  & -74.11            & 28.01            & 50.24\\   
                                           & [-184.14; 35.92]  & [-52.74; 108.76] & [-20.73; 121.21]\\   
         Months relative to app use $=$ 2  & -121.66$^{**}$    & -7.26            & 62.77$^{*}$\\   
                                           & [-228.41; -14.90] & [-85.19; 70.68]  & [-8.57; 134.11]\\   
         Months relative to app use $=$ 3  & -50.48            & -0.26            & 20.39\\   
                                           & [-155.51; 54.56]  & [-81.18; 80.66]  & [-48.99; 89.77]\\   
         Months relative to app use $=$ 4  & -94.18$^{*}$      & 37.26            & 95.80$^{**}$\\   
                                           & [-198.96; 10.59]  & [-46.32; 120.84] & [20.47; 171.13]\\   
         Months relative to app use $=$ 5  & -57.21            & 100.83$^{**}$    & 145.05$^{***}$\\   
                                           & [-148.40; 33.97]  & [18.23; 183.44]  & [67.11; 223.00]\\   
         Month income                      & 0.08$^{***}$      & 0.06$^{***}$     & -0.01\\   
                                           & [0.05; 0.10]      & [0.04; 0.09]     & [-0.03; 0.02]\\   
         Month spend                       & -0.16$^{***}$     & 0.10$^{***}$     & 0.25$^{***}$\\   
                                           & [-0.19; -0.14]    & [0.08; 0.12]     & [0.23; 0.26]\\   
         Active accounts                   & 72.05$^{***}$     & 306.25$^{***}$   & 243.69$^{***}$\\   
                                           & [56.26; 87.84]    & [286.09; 326.40] & [224.27; 263.11]\\   
         \midrule
         \emph{Fixed-effects}\\
         User ID                           & Yes               & Yes              & Yes\\  
         Year-month                        & Yes               & Yes              & Yes\\  
         \midrule
         \emph{Fit statistics}\\
         Observations                      & 188,324           & 188,324          & 188,324\\  
         R$^2$                             & 0.04170           & 0.24635          & 0.31097\\  
         Within R$^2$                      & 0.01007           & 0.03612          & 0.07191\\  
         \midrule \midrule
         \multicolumn{4}{l}{\emph{Clustered (User ID) co-variance matrix, 95\% confidence intervals in brackets}}\\
         \multicolumn{4}{l}{\emph{Signif. Codes: ***: 0.01, **: 0.05, *: 0.1}}\\
      \end{tabular}
   \end{threeparttable}
\end{table}






\subsection{Static results}%
\label{sub:static_results}


\begin{table}[htbp]
   \centering
   \tiny
   \begin{threeparttable}[b]
      \caption{\label{tab:static} Static results}
      \begin{tabular}{lcccccccc}
         \tabularnewline \midrule \midrule
         Dependent Variables: & \multicolumn{4}{c}{Net-inflows} & \multicolumn{4}{c}{Discretionary spend}\\
         Model:          & (1)             & (2)            & (3)             & (4)             & (5)              & (6)            & (7)             & (8)\\  
         \midrule
         \emph{Variables}\\
         App use         & 37.93$^{***}$   & 27.14$^{**}$   & 5.60            & -0.08           & 69.09$^{***}$    & 49.57$^{***}$  & 1.05            & -26.59$^{***}$\\   
                         & [11.16; 64.71]  & [0.73; 53.55]  & [-26.57; 37.78] & [-50.70; 50.55] & [63.78; 74.40]   & [41.46; 57.69] & [-12.92; 15.03] & [-38.73; -14.45]\\   
         Month income    & 0.07$^{***}$    & 0.07$^{***}$   & 0.07$^{***}$    & 0.08$^{***}$    & 0.02$^{***}$     & 0.01$^{***}$   & 0.02$^{***}$    & 0.01$^{***}$\\   
                         & [0.06; 0.08]    & [0.05; 0.10]   & [0.05; 0.09]    & [0.05; 0.10]    & [0.02; 0.02]     & [0.01; 0.02]   & [0.02; 0.03]    & [0.00; 0.02]\\   
         Month spend     & -0.12$^{***}$   & -0.16$^{***}$  & -0.12$^{***}$   & -0.16$^{***}$   & 0.16$^{***}$     & 0.12$^{***}$   & 0.16$^{***}$    & 0.12$^{***}$\\   
                         & [-0.13; -0.11]  & [-0.19; -0.14] & [-0.14; -0.10]  & [-0.19; -0.14]  & [0.16; 0.16]     & [0.11; 0.12]   & [0.15; 0.17]    & [0.11; 0.12]\\   
         Active accounts & 40.31$^{***}$   & 75.54$^{***}$  & 38.87$^{***}$   & 73.25$^{***}$   & 45.89$^{***}$    & 85.71$^{***}$  & 42.11$^{***}$   & 74.92$^{***}$\\   
                         & [32.66; 47.95]  & [60.63; 90.45] & [25.50; 52.23]  & [57.69; 88.81]  & [44.37; 47.40]   & [81.21; 90.20] & [39.61; 44.60]  & [70.47; 79.38]\\   
         Intercept       & -7.11           &                &                 &                 & 172.15$^{***}$   &                &                 &   \\   
                         & [-37.67; 23.45] &                &                 &                 & [166.09; 178.21] &                &                 &   \\   
         \midrule
         \emph{Fixed-effects}\\
         User ID         &                 & Yes            &                 & Yes             &                  & Yes            &                 & Yes\\  
         Year-month      &                 &                & Yes             & Yes             &                  &                & Yes             & Yes\\  
         \midrule
         \emph{Fit statistics}\\
         Observations    & 188,324         & 188,324        & 188,324         & 188,324         & 188,324          & 188,324        & 188,324         & 188,324\\  
         R$^2$           & 0.00888         & 0.04113        & 0.00940         & 0.04163         & 0.41618          & 0.66171        & 0.42425         & 0.67060\\  
         Within R$^2$    &                 & 0.01014        & 0.00877         & 0.01000         &                  & 0.28046        & 0.40985         & 0.23849\\  
         \midrule \midrule
         \multicolumn{9}{l}{\emph{Signif. Codes: ***: 0.01, **: 0.05, *: 0.1}}\\
      \end{tabular}
   \end{threeparttable}
\end{table}






\subsection{Additional TWFE specifications}%
\label{sub:additional_twfe_specifications}

\begin{figure}[H]
    \centering
    \caption{Alternative model specifications}%
    \includegraphics[width=.7\textwidth]{\figdir/dspend_alt.png}
\end{figure}


\begin{table}[htbp]
   \centering
   \tiny
   \begin{threeparttable}[b]
      \caption{\label{tab:dspend_alt} Alternative model specifications}
      \begin{tabular}{lcccc}
         \tabularnewline \midrule \midrule
         Dependent Variable: & \multicolumn{4}{c}{Discretionary spend}\\
         Model:                            & (1)                & (2)                & (3)              & (4)\\  
         \midrule
         \emph{Variables}\\
         Months relative to app use $=$ -6 & -122.89$^{***}$    & -116.38$^{***}$    & -81.19$^{***}$   & -61.31$^{***}$\\   
                                           & [-137.26; -108.52] & [-129.14; -103.61] & [-93.92; -68.47] & [-78.13; -44.48]\\   
         Months relative to app use $=$ -5 & -75.65$^{***}$     & -70.88$^{***}$     & -56.47$^{***}$   & -47.06$^{***}$\\   
                                           & [-94.93; -56.37]   & [-85.23; -56.54]   & [-73.72; -39.23] & [-62.64; -31.47]\\   
         Months relative to app use $=$ -4 & -53.18$^{***}$     & -51.02$^{***}$     & -32.49$^{***}$   & -26.61$^{***}$\\   
                                           & [-72.45; -33.91]   & [-64.94; -37.10]   & [-47.26; -17.72] & [-41.13; -12.09]\\   
         Months relative to app use $=$ -3 & -27.25$^{***}$     & -25.50$^{***}$     & -12.83           & -7.89\\   
                                           & [-46.51; -7.98]    & [-39.08; -11.91]   & [-29.37; 3.72]   & [-21.71; 5.94]\\   
         Months relative to app use $=$ -2 & -6.28              & -5.15              & -3.49            & -1.01\\   
                                           & [-25.55; 12.98]    & [-18.28; 7.98]     & [-20.95; 13.98]  & [-13.00; 11.97]\\   
         Months relative to app use $=$ 0  & -47.60$^{***}$     & -54.52$^{***}$     & -33.64$^{***}$   & -39.76$^{***}$\\   
                                           & [-66.86; -28.34]   & [-67.92; -41.12]   & [-49.51; -17.76] & [-52.94; -26.58]\\   
         Months relative to app use $=$ 1  & -46.13$^{***}$     & -50.48$^{***}$     & -39.90$^{***}$   & -44.23$^{***}$\\   
                                           & [-65.40; -26.86]   & [-64.31; -36.65]   & [-57.33; -22.47] & [-58.30; -30.15]\\   
         Months relative to app use $=$ 2  & -32.80$^{***}$     & -35.47$^{***}$     & -39.02$^{***}$   & -43.24$^{***}$\\   
                                           & [-52.07; -13.53]   & [-49.73; -21.20]   & [-63.52; -14.51] & [-58.23; -28.24]\\   
         Months relative to app use $=$ 3  & -29.28$^{***}$     & -31.36$^{***}$     & -45.09$^{***}$   & -49.52$^{***}$\\   
                                           & [-48.56; -10.01]   & [-45.57; -17.15]   & [-64.44; -25.75] & [-65.30; -33.74]\\   
         Months relative to app use $=$ 4  & -26.75$^{***}$     & -28.93$^{***}$     & -40.10$^{***}$   & -44.45$^{***}$\\   
                                           & [-46.02; -7.47]    & [-43.25; -14.60]   & [-65.31; -14.89] & [-61.37; -27.53]\\   
         Months relative to app use $=$ 5  & -13.23$^{*}$       & -23.94$^{***}$     & -38.01$^{***}$   & -56.39$^{***}$\\   
                                           & [-27.83; 1.37]     & [-36.72; -11.16]   & [-55.92; -20.11] & [-76.52; -36.27]\\   
         Month income                      & 0.02$^{***}$       & 0.01$^{***}$       & 0.02$^{***}$     & 0.01$^{***}$\\   
                                           & [0.02; 0.03]       & [0.01; 0.02]       & [0.02; 0.03]     & [0.00; 0.02]\\   
         Month spend                       & 0.16$^{***}$       & 0.12$^{***}$       & 0.16$^{***}$     & 0.12$^{***}$\\   
                                           & [0.16; 0.16]       & [0.11; 0.12]       & [0.15; 0.17]     & [0.11; 0.12]\\   
         Active accounts                   & 41.94$^{***}$      & 77.85$^{***}$      & 40.80$^{***}$    & 72.39$^{***}$\\   
                                           & [40.39; 43.49]     & [73.30; 82.41]     & [38.37; 43.22]   & [67.89; 76.88]\\   
         Intercept                         & 274.97$^{***}$     &                    &                  &   \\   
                                           & [260.03; 289.91]   &                    &                  &   \\   
         \midrule
         \emph{Fixed-effects}\\
         User ID                           &                    & Yes                &                  & Yes\\  
         Year-month                        &                    &                    & Yes              & Yes\\  
         \midrule
         \emph{Fit statistics}\\
         Observations                      & 188,324            & 188,324            & 188,324          & 188,324\\  
         R$^2$                             & 0.41828            & 0.66337            & 0.42491          & 0.67097\\  
         Within R$^2$                      &                    & 0.28398            & 0.41052          & 0.23935\\  
         \midrule \midrule
         \multicolumn{5}{l}{\emph{Signif. Codes: ***: 0.01, **: 0.05, *: 0.1}}\\
      \end{tabular}
   \end{threeparttable}
\end{table}





\begin{figure}[H]
    \centering
    \caption{Alternative model specifications}%
    \includegraphics[width=.7\textwidth]{\figdir/netflows_alt.png}
\end{figure}


\begin{table}[htbp]
   \centering
   \tiny
   \begin{threeparttable}[b]
      \caption{\label{tab:netflows_alt} Alternative model specifications}
      \begin{tabular}{lcccc}
         \tabularnewline \midrule \midrule
         Dependent Variable: & \multicolumn{4}{c}{Net-inflows}\\
         Model:                            & (1)               & (2)               & (3)               & (4)\\  
         \midrule
         \emph{Variables}\\
         Months relative to app use $=$ -6 & -112.73$^{***}$   & -104.53$^{**}$    & -104.14$^{**}$    & -104.79$^{**}$\\   
                                           & [-185.30; -40.15] & [-187.23; -21.82] & [-189.54; -18.75] & [-197.45; -12.13]\\   
         Months relative to app use $=$ -5 & -44.39            & -41.53            & -45.12            & -45.90\\   
                                           & [-141.77; 52.99]  & [-148.42; 65.37]  & [-120.66; 30.41]  & [-154.47; 62.67]\\   
         Months relative to app use $=$ -4 & -91.89$^{*}$      & -91.29$^{*}$      & -84.19            & -86.20\\   
                                           & [-189.22; 5.44]   & [-199.28; 16.69]  & [-200.74; 32.37]  & [-195.61; 23.21]\\   
         Months relative to app use $=$ -3 & -94.20$^{*}$      & -93.45$^{*}$      & -92.07$^{*}$      & -92.58\\   
                                           & [-191.50; 3.10]   & [-203.59; 16.70]  & [-188.41; 4.27]   & [-203.84; 18.67]\\   
         Months relative to app use $=$ -2 & -118.68$^{**}$    & -118.36$^{**}$    & -119.19$^{***}$   & -119.53$^{**}$\\   
                                           & [-215.98; -21.39] & [-222.44; -14.29] & [-200.97; -37.41] & [-223.87; -15.19]\\   
         Months relative to app use $=$ 0  & -43.44            & -50.00            & -33.48            & -37.36\\   
                                           & [-140.74; 53.85]  & [-162.78; 62.77]  & [-127.72; 60.75]  & [-150.33; 75.61]\\   
         Months relative to app use $=$ 1  & -66.94            & -71.74            & -74.79            & -74.11\\   
                                           & [-164.24; 30.37]  & [-180.80; 37.33]  & [-184.62; 35.05]  & [-184.14; 35.92]\\   
         Months relative to app use $=$ 2  & -115.51$^{**}$    & -118.47$^{**}$    & -125.25$^{**}$    & -121.66$^{**}$\\   
                                           & [-212.82; -18.19] & [-224.24; -12.70] & [-224.00; -26.50] & [-228.41; -14.90]\\   
         Months relative to app use $=$ 3  & -43.50            & -45.81            & -56.03            & -50.48\\   
                                           & [-140.83; 53.83]  & [-149.17; 57.54]  & [-161.02; 48.97]  & [-155.51; 54.56]\\   
         Months relative to app use $=$ 4  & -90.18$^{*}$      & -92.22$^{*}$      & -101.71$^{*}$     & -94.18$^{*}$\\   
                                           & [-187.52; 7.17]   & [-194.27; 9.84]   & [-208.92; 5.49]   & [-198.96; 10.59]\\   
         Months relative to app use $=$ 5  & -51.70            & -52.18            & -72.25            & -57.21\\   
                                           & [-125.43; 22.03]  & [-133.90; 29.54]  & [-161.36; 16.86]  & [-148.40; 33.97]\\   
         Month income                      & 0.07$^{***}$      & 0.07$^{***}$      & 0.07$^{***}$      & 0.08$^{***}$\\   
                                           & [0.06; 0.08]      & [0.05; 0.10]      & [0.05; 0.09]      & [0.05; 0.10]\\   
         Month spend                       & -0.12$^{***}$     & -0.16$^{***}$     & -0.12$^{***}$     & -0.16$^{***}$\\   
                                           & [-0.13; -0.11]    & [-0.19; -0.14]    & [-0.14; -0.10]    & [-0.19; -0.14]\\   
         Active accounts                   & 38.47$^{***}$     & 72.54$^{***}$     & 37.00$^{***}$     & 72.05$^{***}$\\   
                                           & [30.62; 46.31]    & [56.98; 88.10]    & [24.59; 51.40]    & [56.26; 87.84]\\   
         Intercept                         & 95.97$^{**}$      &                   &                   &   \\   
                                           & [20.52; 171.42]   &                   &                   &   \\   
         \midrule
         \emph{Fixed-effects}\\
         User ID                           &                   & Yes               &                   & Yes\\  
         Year-month                        &                   &                   & Yes               & Yes\\  
         \midrule
         \emph{Fit statistics}\\
         Observations                      & 188,324           & 188,324           & 188,324           & 188,324\\  
         R$^2$                             & 0.00897           & 0.04121           & 0.00948           & 0.04170\\  
         Within R$^2$                      &                   & 0.01022           & 0.00885           & 0.01007\\  
         \midrule \midrule
         \multicolumn{5}{l}{\emph{Signif. Codes: ***: 0.01, **: 0.05, *: 0.1}}\\
      \end{tabular}
   \end{threeparttable}
\end{table}






\subsection{Subgroups}%
\label{sub:subgroups}

To analyse which groups benefit most from adopting Money Dashboard, we split
our sample by gender, generation, income quartiles, and pre-adoption savings
behaviour.

We define generations as follows: boomers were born between 1946 and 1964, Gen
X between 1965 and 1980, Millennials between 1981 and 1996, and Gen Z after
1997.\footnote{Based on age ranges provides by
    \href{https://www.beresfordresearch.com/age-range-by-generation/}{Beresford
Research}.}

Subgroup analysis: same Fig an Tab as in main analysis, but with line for each
subgroup. One figure for each of: gender, generations, income terciles,
per-adoption average savings tercile (inspired by \citet{carlin2017fintech},
see Fig 5 and Table 4).

See also section 6 in \citet{gargano2021goal}


