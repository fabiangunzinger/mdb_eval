% !TEX root = ../eval.tex

\begin{abstract}

In this paper, I test whether using Money Dashboard is associated with a
reduction in discretionary spending and an increase in ``emergency savings''. I
find that users reduce their discretionary spend by between \pounds100 and
\pounds150 (11-17\% of average discretionary spend) once they start using the
app and sustain that reduction throughout the six-month post-signup period I
consider. In contrast, I cannot find an increase in short-term or long-term
savings. Looking at disaggregated measures of discretionary spend further
suggests that the reduction in spend is the result of maintained month-to-month
changes in behaviour rather than one-off cancellations of direct-debit
transactions, that it results from reductions across a number of spending
categories, and that it is a result of changes along the extensive rather than
the intensive margin -- users reduce the number of transactions they make
rather than the average transaction value.

\end{abstract}

