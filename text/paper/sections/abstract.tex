% !TEX root = ../eval.tex

\begin{abstract}

In this paper, I test whether using Money Dashboard is associated with a
reduction in discretionary spending and an increase in ``rainy-day savings''. I
find that users reduce their discretionary spend by between \pounds100 and
\pounds150 (11-17\% of average discretionary spend) once they start using the
app and sustain that reduction throughout the six-month post-signup period I
consider. Looking at disaggregated measures of discretionary spend further
shows that the reduction is the result of maintained month-to-month changes in
behaviour rather than one-off cancellations of direct-debit transactions, that
it results from reducing spending on a number of different categories of
purchases rather than a single one, and that it is a result of changes along
the extensive rather than the intensive margin -- users reduce the number of
transactions they make rather than the value of the average transaction.
Interestingly, users do not seem to use these additional funds to build up
``rainy-day savings'': net-inflows into savings accounts do not change after
signup. I can also neither find significant increases in flows into investment
and pension accounts or additional savings accounts that are not linked to the
app, not additional loan repayments.

\end{abstract}

